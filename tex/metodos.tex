%--------------------------------------------------------------------------------------
% Este arquivo contém a sua metodologia
%--------------------------------------------------------------------------------------
\chapter{Metodologia} \label{ch:MM} %Uma label é como você referencia uma seção no texto com a tag \ref{}
Neste capítulo, será apresentada a metodologia aplicada para o desenvolvimento deste produto do Trabalho de Conclusão de Curso (TCC), que possui natureza técnica 
e tem como objetivo o desenvolvimento de um sistema integrado de travamento de portas veicular, quais as ferramentas utilizadas e como o Desenvolvimento Baseado 
em Comportamento (BDD) foi empregado.

\section{\textbf{Tipo de Pesquisa}} 
Esta é uma pesquisa aplicada, tendo em vista à solução de problemas presentes em metodologias de desenvolvimento de produto tradicionais como rastreabilidade 
de requisitos e aplicação de testes manuais. Também será aplicada uma abordagem mista, que demonstra quantitativamente a cobertura dos testes desenvolvidos em 
Modelo em Loop (MIL) e qualitativamente na experiência de usuário no modelo final.

\section{\textbf{Procedimentos Metodológicos}}
Este produto teve seu desenvolvimento em etapas com base na metodologia do processo de BDD como demonstrado em \citeonline{studyBDD}, com uma série de adaptações específicas 
para o contexto da Engenharia Automotiva. As etapas de desenvolvimento a seguir serão aplicadas:
\begin{enumerate}
    \item Descrição das histórias de usuário: Capturar as histórias definidas em forma de requisitos funcionais com linguagem natural e que capture o valor 
    gerado pela perspectiva do usuário;
    \item Mapeamento de exemplos: Definição de exemplos concretos tomados da perspectiva do usuário final;
    \item Desenho do diagrama de caixa preta: Desenho do diagrama do sistema que demonstra suas interfaces de entrada e saída, sem demonstrar detalhes de suas interações;
    \item Desenvolvimento dos cenários Gherkin: Escrita dos cenários aplicando o padrão cucumber e que aborde todas as regras definidas;
    \item Tradução dos cenários em testes: Desenvolvimento de funções dos passos dos cenários gherkin utilizando as interfaces do diagrama de caixa branca;
    \item Modelagem iterativa do sistema em simulink: Modelagem feita de forma iterativa aplicando os cenários como critério de aceitação a cada iteração;
    \item Análise qualitativa do produto final: Testes de uso do produto final e aprovação da experiência de usuário entregue.
\end{enumerate}

Neste projeto, algumas etapas serão repetidas, pois o desenvolvimento orientado por comportamento (BDD) é um processo incremental, baseado em iterações 
sucessivas que introduzem novas funcionalidades ao produto. Assim, as etapas 1 e 2 serão inicialmente executadas para cada história de usuário. Ao final 
desse ciclo, a etapa 3 será iniciada elaborando um diagrama de caixa preta que abrange todas as funcionalidades definidas até o momento.

\section{\textbf{Ferramentas Utilizadas}}
As ferramentas e tecnologias adotadas no desenvolvimento deste produto foram escolhidas com base na compatibilidade com o modelo proposto e na sua capacidade 
de integrar diferentes etapas do processo. A seguir, descreve-se cada uma delas:

\begin{itemize}
    \item \textbf{Cucumber (Gherkin)}: utilizado como padrão para a escrita de cenários comportamentais, permitindo a especificação dos requisitos no formato Given-When-Then, de forma legível por humanos e máquinas;
    \item \textbf{Miro}: empregado para a criação de diagramas não técnicos, bem como para auxiliar na discussão colaborativa de exemplos, fluxos e histórias de usuário;
    \item \textbf{Visual Studio Code}: ambiente de desenvolvimento (IDE) utilizado para escrever os cenários em Gherkin, desenvolver a tradução para testes executáveis e integrar o código gerado ao microcontrolador;
    \item \textbf{Python}: linguagem escolhida para implementar a lógica de tradução dos cenários Gherkin em testes executáveis, permitindo automatização e validação do comportamento esperado do sistema.
    \begin{itemize}
        \item \textbf{Behave}: biblioteca utilizada para interpretar e executar os cenários comportamentais no formato BDD, integrando as especificações Gherkin à lógica de teste.
        \item \textbf{Matlab Engine API for Python}: biblioteca usada para permitir a comunicação entre scripts Python e o ambiente Simulink, possibilitando a execução dos testes durante a simulação.
    \end{itemize}    
    \item \textbf{Simulink}: ferramenta adotada para a modelagem do sistema embarcado funcional, possibilitando simulações e geração automática de código C para o sistema-alvo.
    \item \textbf{Git/Github}: utilizado para o versionamento do projeto que inclui o modelo simulink do sistema, código embarcado gerado do modelo, arquivos feature dos cenários gherkin e códigos python dos testes executáveis.

\end{itemize}

\section{\textbf{Critérios de Avaliação}}
Para a verificação e validação do produto gerado, serão aplicados critérios quantitativos de análise, com base nos testes de aceitação de cada história de usuário. Este 
processo será realizado durante a etapa 6, correspondente à modelagem iterativa do sistema no Simulink.

A cada iteração, os testes de aceitação de uma história de usuário serão executados e, com base nos cenários não aprovados, serão identificadas falhas no comportamento 
do modelo. A partir dessas falhas, serão implementadas soluções para garantir a aprovação dos testes que podem envolver ajustes na lógica do modelo, 
alterações na escrita dos cenários Gherkin ou atualizações nas funções de definição de passos.

Após a aprovação dos testes de uma história, todos os cenários previamente aprovados serão reexecutados. Dessa forma, é possível verificar que modificações realizadas 
para atender a um cenário específico não impactaram os demais, garantindo a robustez do sistema.

Ao final do processo, todos os cenários de todas as histórias de usuário serão executados em conjunto, assegurando que o produto final atenda a todas as funcionalidades 
mapeadas e que 100\% dos testes de aceitação sejam aprovados.

%--------------------------------------------------------------------------------------
% Insere a seção de cronograma
% Está comentada porque só é necessária no TCC I
%--------------------------------------------------------------------------------------

%\section{Cronograma} \label{sec:crono}

%A tabela \ref{tab:cronograma} mostra o cronograma de atividades a serem executadas para o TCC II, com base no calendário de 201X.Y da UNIVASF.

%\newpage
%\begin{table}[!thb]
%	%\huge
%    \centering
%    \caption{\label{tab:cronograma} Cronograma das atividades previstas para o TCC II}
%%    \begin{adjustbox}{max width=\textwidth}
%    \begin{tabular}{p{6.5cm}|c|c|c|c|c|c}
%    \toprule
%    \textbf{Atividade}                      & Nov & Dez & Jan & Fev & Mar & Abr \\ \hline
%    Implementar o banco de dados              & X    & X     &       &        &          &          \\ \hline
%    Desenvolver a API HTTP RESTful                      &   X   & X     &       &        &          &          \\ \hline
%    Implementar o serviço de captura de dados        &      &      & X     &   X     &          &          \\ \hline
%    Desenvolver a aplicação \textit{Web/mobile} para exibição dos dados         &      &      & X     &   X     &     X     &          \\ \hline
 %   Teste do sistema            &      &       &       &        & X        &          %\\ \hline
 %   Escrita do TCC II                       &   X   & X     & X     & X      & X        & X        \\ \hline
%   Defesa do TCC II                        &      &       &       &        &          & X       \\
%    \bottomrule
 %   \end{tabular}
 %   \end{adjustbox}
%    \legend{\textbf{Fonte:} O autor.}
%\end{table}

