%--------------------------------------------------------------------------------------
% Este arquivo contém a sua metodologia
%--------------------------------------------------------------------------------------
\chapter{Metodologia} \label{ch:MM} %Uma label é como você referencia uma seção no texto com a tag \ref{}
Neste capítulo, será apresentada a metodologia aplicada para o desenvolvimento deste produto do Trabalho de Conclusão de Curso (TCC), que possui natureza técnica 
e tem como objetivo a adapatação do processo de \textit{Behavior-Driven Development} (BDD) ao contexto da Engenharia Automotiva. A prova de conceito é realizada a 
partir da aplicação da metodologia na produção de um sistema de travamento de portas veicular. 

Esta é uma pesquisa aplicada, voltada à solução de complexidades presentes em metodologias de desenvolvimento de produto tradicionais como a rastreabilidade de 
requisitos e a execução de testes manuais. Para a avaliação dos resultados, foi aplicada uma abordagem quantitativa, com a finalidade de metrificar a cobertura 
dos testes de caixa-preta desenvolvidos sobre o modelo do sistema, além de identificar falhas que guiam a sua implementação. 

\section{\textbf{Procedimentos Metodológicos}}
Este produto teve seu desenvolvimento com base na metodologia do processo de BDD como demonstrado em \citeonline{studyBDD}, com uma série de adaptações específicas 
para o contexto da Engenharia Automotiva, resultando nas seguintes etapas:
\begin{enumerate}
    \item Descrição das histórias de usuário: Capturar as histórias definidas em forma de requisitos funcionais com linguagem natural e que capture o valor 
    gerado pela perspectiva do usuário;
    \item Mapeamento de exemplos: Definição de exemplos concretos tomados da perspectiva do usuário final;
    \item Desenho do diagrama de caixa preta: Desenho do diagrama do sistema que demonstra suas interfaces de entrada e saída, sem demonstrar detalhes de suas interações;
    \item Desenvolvimento dos cenários Gherkin: Escrita dos cenários aplicando o padrão cucumber e que aborde todas as regras definidas;
    \item Modelagem iterativa do sistema em simulink: Modelagem feita de forma iterativa com a criação das definições de passos dos cenários, seguido da sua aplicação como critérios de aceitação;
    \item 
\end{enumerate}

Durante a primeira etapa (\ref{sbs:etapa1}) foram criadas as histórias de usuário que definem as funcionalidades a serem implementadas pelo sistema, 
pela perspectiva do usuário. Para isso, a base utilizada se utiliza da bibliografia disponível para descrever como veículos implementam o travamento das portas, 
mais especificamente tomando exemplos dos veículos da Volkswagen \cite{vwLocking} e da Bosch \cite{bosch2022handbook,reif2017locking}.

Após a criação das histórias de usuário, a segunda etapa (\ref{sbs:etapa2}) tem foco na aplicação do mapeamento de exemplos que incrementa o design do sistema definindo aspectos 
que descrevem a utilização da funcionalidade pela perspectiva do usuário. Para isso, os seguintes pontos foram definidos:

\begin{itemize}
    \item Exemplos concretos de uso da funcionalidade pela perspectiva do usuário;
    \item Regras de comportamento que clarificam a responsta esperada pelo sistema nos exemplos citados;
    \item Perguntas que não são pertinentes para a definição dos exemplos, sendo respondidas ao fim da definição da história.
\end{itemize}

As perguntas geradas na definição dos exemplos poderiam ser respondidas no fim do processo, tendo em vista que, a princípio, elas não afetam a definição 
dos comportamentos do sistema. No entanto, optou-se por tratá-las antes de prosseguir para as próximas histórias, para garantir que haja o entendimento 
mais detalhado do comportamento esperado e possibilitar a evolução incremental das funcionalidades definidas.

Na terceira etapa (\ref{sbs:etapa3}), um diagrama do sistema em formato de caixa-preta foi criado, para suportar a criação dos cenários \textit{Gherkin}. Ele 
também serve como uma base para a modelagem em Simulink, especificando as entradas e saídas que serão utilizados na execução dos testes.

Em seguida, são criados os cenários \textit{Gherkin} na quarta etapa (\ref{sbs:etapa4}), que definem o comportamento do sistema e atuam como testes de aceitação 
da história. Cada um dos exemplos levantados durante o mapeamento deve ser coberto por um ou mais cenários, para garantir que as situações de teste cubram 
a interação prevista.

Durante a quinta etapa (\ref{sbs:etapa5}), será feita a modelagem iterativa do sistema, baseada na execução dos testes de aceitação dos cenários. Este processo será 
feito com a repetição dos seguinte passos:

\begin{enumerate}
    \item Execução dos cenários de uma das histórias de usuário a partir do arquivo \textbf{.feature};
    \item Identificação dos erros obtidos a partir do relatório de testes;
    \item Implementação da solução no modelo, passos ou definição dos passos;
    \item Repetição dos passos 1, 2 e 3 para que todas as falhas sejam identificadas e todos os testes da história sejam aprovados;
    \item Aprovação da história (caso todos os cenários tenham sido aprovados);
    \item Repetição dos testes das histórias aprovadas anteriormente para constatar possíveis falhas geradas na implementação de novos comportamentos;
    \item Identificação e implementação de soluções das falhas das histórias de usuário anteriores;
    \item Repetição dos passos para a história seguinte.
\end{enumerate}

Desta maneira, a execução dos testes é sempre realizada de maneira repetitiva, para garantir que a evolução gradual do sistema seja alcançada sem afetar 
nenhum comportamento aprovado previamente. Esta constatação repetida traz um maior nível de cobertura de testes, pois possibilita identificar possíveis 
conflitos gerados entre as diferentes histórias de usuário e contribui para a maior robustez do sistema, dos cenários e das definições de passo.

Ao final da modalgem iterativa, o conjunto completo de todos os cenários de todas as histórias de usuário deve ser executado. Isso é feito como uma 
validação final de que todo o comportamento definido para o sistema é satisfeito.

Por fim, na sexta etapa (\ref{sbs:etapa6}), será feita uma análise quantitativa a cerca dos resultados obtidos durante a aplicação do processo. Serão 
demonstradas instâncias de falhas de teste capturadas por conta da aplicação da metodologia, assim como a maior cobertura de testes foi resultante das 
etapas de análise.

\section{\textbf{Ferramentas Utilizadas}}
As ferramentas e tecnologias adotadas no desenvolvimento deste produto foram escolhidas com base na compatibilidade com o modelo proposto e na sua capacidade 
de integrar diferentes etapas do processo. A seguir, descreve-se cada uma delas:

\begin{itemize}
    \item \textbf{Cucumber (Gherkin)}: utilizado como padrão para a escrita de cenários comportamentais, permitindo a especificação dos requisitos no formato Given-When-Then, de forma legível por humanos e máquinas;
    \item \textbf{Miro}: website empregado para a criação de diagramas não técnicos, bem como para auxiliar na discussão colaborativa de exemplos, fluxos e histórias de usuário;
    \item \textbf{Visual Studio Code}: ambiente de desenvolvimento (IDE) utilizado para escrever os cenários em Gherkin, desenvolver a tradução para testes executáveis e integrar o código gerado ao microcontrolador;
    \item \textbf{Python}: linguagem escolhida para implementar a lógica de tradução dos cenários Gherkin em testes executáveis, permitindo automatização e validação do comportamento esperado do sistema.
    \begin{itemize}
        \item \textbf{Behave}: biblioteca utilizada para interpretar e executar os cenários comportamentais no formato BDD, integrando as especificações Gherkin à lógica de teste.
        \item \textbf{Matlab Engine API for Python}: biblioteca usada para permitir a comunicação entre scripts Python e o ambiente Simulink, possibilitando a execução dos testes durante a simulação.
    \end{itemize}    
    \item \textbf{Simulink}: ferramenta adotada para a modelagem do sistema embarcado funcional, possibilitando simulações e geração automática de código C para o sistema-alvo.
    \item \textbf{Git/Github}: utilizado para o versionamento do projeto que inclui o modelo simulink do sistema, código embarcado gerado do modelo, arquivos feature dos cenários gherkin e códigos python dos testes executáveis.

\end{itemize}

\section{\textbf{Critérios de Avaliação}}
Para a verificação e validação do produto gerado, serão aplicados critérios quantitativos de análise, com base nos testes de aceitação de cada história de usuário. Este 
processo será realizado durante a etapa 6, correspondente à modelagem iterativa do sistema no Simulink.

A cada iteração, os testes de aceitação de uma história de usuário serão executados e, com base nos cenários não aprovados, serão identificadas falhas no comportamento 
do modelo. A partir dessas falhas, serão implementadas soluções para garantir a aprovação dos testes que podem envolver ajustes na lógica do modelo, 
alterações na escrita dos cenários Gherkin ou atualizações nas funções de definição de passos.

Após a aprovação dos testes de uma história, todos os cenários previamente aprovados serão reexecutados. Dessa forma, é possível verificar que modificações realizadas 
para atender a um cenário específico não impactaram os demais, garantindo a robustez do sistema.

Ao final do processo, todos os cenários de todas as histórias de usuário serão executados em conjunto, assegurando que o produto final atenda a todas as funcionalidades 
mapeadas e que 100\% dos testes de aceitação sejam aprovados.

%--------------------------------------------------------------------------------------
% Insere a seção de cronograma
% Está comentada porque só é necessária no TCC I
%--------------------------------------------------------------------------------------

%\section{Cronograma} \label{sec:crono}

%A tabela \ref{tab:cronograma} mostra o cronograma de atividades a serem executadas para o TCC II, com base no calendário de 201X.Y da UNIVASF.

%\newpage
%\begin{table}[!thb]
%	%\huge
%    \centering
%    \caption{\label{tab:cronograma} Cronograma das atividades previstas para o TCC II}
%%    \begin{adjustbox}{max width=\textwidth}
%    \begin{tabular}{p{6.5cm}|c|c|c|c|c|c}
%    \toprule
%    \textbf{Atividade}                      & Nov & Dez & Jan & Fev & Mar & Abr \\ \hline
%    Implementar o banco de dados              & X    & X     &       &        &          &          \\ \hline
%    Desenvolver a API HTTP RESTful                      &   X   & X     &       &        &          &          \\ \hline
%    Implementar o serviço de captura de dados        &      &      & X     &   X     &          &          \\ \hline
%    Desenvolver a aplicação \textit{Web/mobile} para exibição dos dados         &      &      & X     &   X     &     X     &          \\ \hline
 %   Teste do sistema            &      &       &       &        & X        &          %\\ \hline
 %   Escrita do TCC II                       &   X   & X     & X     & X      & X        & X        \\ \hline
%   Defesa do TCC II                        &      &       &       &        &          & X       \\
%    \bottomrule
 %   \end{tabular}
 %   \end{adjustbox}
%    \legend{\textbf{Fonte:} O autor.}
%\end{table}

