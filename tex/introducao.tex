%--------------------------------------------------------------------------------------
% Este arquivo contém a sua introdução, objetivos e organização do trabalho
%--------------------------------------------------------------------------------------
\chapter{Introdução}

A indústria automotiva tem experimentado um crescimento constante, acompanhado por um aumento significativo na modernização dos veículos e pela intensa competição entre 
montadoras para entregar produtos cada vez mais robustos e de alta qualidade. Esse cenário traz consigo um aumento na complexidade do desenvolvimento, demandando soluções 
mais eficientes e seguras para sistemas cada vez mais sofisticados e que sejam capazes de atender às necessidades específicas de cada cliente.

A modernização do setor automotivo tem conduzido ao próximo grande avanço: os veículos definidos por software (\textit{Software-Defined Vehicles}) 
\cite{softwareVehicles2025}. Essa abordagem possibilita a customização contínua e a evolução do produto mesmo após a aquisição, permitindo a entrega de novas 
funcionalidades ao veículo por meio de atualizações de software feitas remotamente \textit{Over-the-Air (OTA)}, suportadas pelo hardware já embarcado.

À medida que a complexidade dos sistemas aumenta, a capacidade de identificar e corrigir problemas torna-se cada vez mais essencial. Nesse contexto, o denominado 
\textit{Cost of Poor Quality} \cite{costOfPoorQuality} representa um desafio recorrente na indústria automotiva, uma vez que os custos de implementação de soluções 
tendem a crescer significativamente quanto mais tarde no processo de desenvolvimento os erros são detectados. 

Diante desse contexto, a Engenharia Automotiva enfrenta o desafio de projetar e desenvolver produtos com um nível altíssimo de qualidade, ao mesmo tempo em que busca reduzir 
erros e retrabalho. A aplicação de metodologias ágeis, como o \textit{Behavior Driven Development (BDD)} \cite{north2006bdd}, oferece uma abordagem promissora, pois foca na 
definição do comportamento esperado do sistema a partir da perspectiva do usuário. Essa abordagem permite uma colaboração mais efetiva entre os membros de uma equipe ágil 
\cite{atlassianAgileTeams}, facilita a descoberta de funcionalidades e possibilita que o desenvolvimento seja guiado por testes de aceitação automatizados, promovendo 
entregas de valor de forma mais rápida e com maior confiabilidade.

\section{\textbf{Justificativa}}

O problema central deste trabalho é: ``como é possível aplicar o processo de BDD para facilitar a definição do sistema e guiar seu desenvolvimento no contexto da 
Engenharia de Sistemas Automotivos?''. A solução proposta consiste na adaptação do processo, integrando as técnicas já consolidadas do BDD com metodologias 
amplamente utilizadas na indústria automotiva, de forma a lidar com as complexidades inerentes ao desenvolvimento de sistemas veiculares.

Este trabalho insere-se no contexto atual da indústria automotiva, adotando o sistema de travamento de portas como prova de conceito para a geração do produto. 
Durante seu desenvolvimento, serão consideradas como referência implementações reais de sistemas já presentes em veículos, como os da Volkswagen \cite{vwLocking} 
e da Bosch \cite{bosch2022handbook,reif2017locking}. Nesse sentido, a adaptação do processo de BDD será aplicada ao desenvolvimento de um produto compatível com 
as especificações industriais, ao mesmo tempo em que busca minimizar as complexidades enfrentadas em sua execução.

O BDD incorpora diversas premissas que surgem das lições aprendidas na aplicação de outras metodologias ágeis, definindo o desenvolvimento do produto a partir do 
comportamento que gera valor ao usuário. Além disso, ele busca promover a colaboração entre as diferentes etapas do desenvolvimento, garantindo que toda a equipe 
compreenda o produto e consiga assumir a perspectiva do usuário.

Por outro lado, a Engenharia Automotiva apresenta uma complexidade significativa, o que torna desafiadora a entrega de produtos de alta qualidade. A definição de 
um sistema veicular envolve muito mais do que software ou eletrônica, sendo necessário integrar componentes mecânicos, considerar falhas 
potenciais, garantir a rastreabilidade de requisitos e contemplar casos de uso extremos.

O principal desafio deste trabalho consiste em identificar técnicas capazes de capturar as complexidades envolvidas no desenvolvimento de sistemas automotivos 
e adaptá-las às premissas do BDD. Em casos como o do sistema de travamento de portas, isso é traduzido na necessidade de atender a funcionalidades 
essenciais, como \cite{bosch2022handbook}:

\begin{itemize}
    \item Permitir o acesso ao interior do veículo para todas as pessoas autorizadas;
    \item Gerenciar a abertura das portas a partir das maçanetas internas e externas;
    \item Proteger os ocupantes da abertura indevida das portas;
    \item Proteger o veículo contra furtos enquanto ele está estacionado.
\end{itemize}

Uma das principais complexidades desse tipo de sistema está na definição do mecanismo empregado no travamento das portas. Conforme descrito por 
\citeonline{reif2017locking}, podem ser utilizadas travas mecânicas ou eletrônicas, cada um com conjuntos específicos de componentes e lógicas de 
software distintas que influenciam diretamente o desenvolvimento. Esse desafio é recorrente na indústria automotiva, na qual diversos sistemas dependem 
de decisões técnicas de hardware ou soluções mecânicas para viabilizar o início do processo de design.

Neste trabalho, será apresentada uma metodologia que demonstra como a aplicação do Desenvolvimento Orientado a Comportamento (BDD) possibilita projetar o 
sistema de forma independente dos detalhes técnicos, facilitando tanto o desenvolvimento das funcionalidades quanto a execução dos testes. Isso ocorre 
porque a abordagem do BDD prioriza a perspectiva do usuário e a especificação do comportamento esperado do sistema, os quais não estão diretamente vinculados 
à implementação.

\section{\textbf{Objetivos}}

\subsection{Objetivo Geral}
%Para responder à problemática levantada, o presente estudo tem como objetivo geral
Adaptar a metodologia de \textit{Behavior Driven Development} (BDD) ao contexto do desenvolvimento de sistemas automotivos, aplicando-a na implementação de um sistema 
de travamento de portas por meio da definição de funcionalidades, da execução de testes de aceitação automatizados e da demonstração de seu potencial para promover um 
processo de desenvolvimento de software automotivo mais organizado e eficiente.

\subsection{Objetivos Específicos}
\begin{itemize}
    \item Detalhar as principais características da metodologia de \textit{Behavior Driven Development} (BDD);
    \item Definir as adaptações necessárias no processo de BDD para torná-lo compatível com a Engenharia Automotiva;
    \item Empregar técnicas de desenvolvimento já consolidadas na indústria automotiva, como o padrão de software AUTOSAR \cite{autosarClassic} e o Design Orientado por Modelos (\textit{Model-Based Design - MBD}) \cite{mathworksMBD2024};
    \item Aplicar os princípios do BDD para a definição das funcionalidades e dos testes de aceitação do sistema de travamento de portas;
    \item Utilizar testes de aceitação automatizados como guia para o desenvolvimento do sistema, garantindo consistência com o design estabelecido;
    \item Implementar o sistema de forma iterativa, possibilitando a identificação e incorporação de melhorias ao longo do processo;
    \item Analisar os resultados obtidos, realizando comparações com metodologias alternativas.
\end{itemize}

% \section{Metodologia}

% A metodologia utilizada neste estudo baseia-se na revisão da bibliografia disponível, com o objetivo de esclarecer o processo de BDD e investigar como ele pode 
% ser adaptado para incorporar práticas consolidadas na Engenharia Automotiva. Para tanto, serão aplicadas técnicas como o mapeamento de exemplos 
% \cite{Lawrence2019cucumber, cucumberExampleMapping}
% , voltado ao levantamento e definição de funcionalidades, e testes de aceitação automatizados 
% \cite{studyBDD}
% , que servirão para guiar o desenvolvimento do sistema.

% A implementação do sistema simulado também fará uso de técnicas típicas da Engenharia Automotiva, incluindo o Model Based Design
% \cite{Vincentelli2001, mathworksMBD2024} e a geração automática de código segundo o padrão de software AUTOSAR 
% \cite{autosarClassic}. Conceitos adicionais do domínio automotivo serão considerados para 
% assegurar a adequada abstração nas diferentes etapas do desenvolvimento.

% Espera-se que os resultados obtidos evidenciam os benefícios da integração do BDD no contexto automotivo, destacando sua contribuição para a qualidade e 
% confiabilidade do sistema. A validação será realizada por meio da execução dos testes de aceitação e da análise do uso do sistema final sob a perspectiva do usuário.

\section{\textbf{Organização do trabalho}}
No Capítulo \ref{ch:te}, serão apresentados os conceitos que compõem a fundamentação teórica deste trabalho, abrangendo:

\begin{itemize}
    \item \textbf{O processo de BDD} - histórico do desenvolvimento do processo e seus pilares fundamentais do processo;
    \item \textbf{Mapeamento de exemplos} - metodologia aplicada para realizar a análise do sistema e definição de funcionalidades;
    \item \textbf{A linguagem \textit{Gherkin}} - linguagem usada para a escrita de cenários, que atuam como definição de comportamentos e como testes de aceitação;
    \item \textbf{Design orientado por modelo} - metodologia adotada para a implementação do sistema;
    \item \textbf{Sistemas de travamento de porta veiculares} - definição e caracterização dos sistemas de segurança voltados ao travamento de portas em veículos. 
\end{itemize}

Em seguida, no Capítulo \ref{ch:MM}, será apresentada a metodologia adotada, descrevendo o procedimento de desenvolvimento do produto, estruturado em seis etapas. 
Serão também detalhadas as ferramentas utilizadas neste trabalho, bem como os critérios de avaliação, definidos a partir da execução dos testes de aceitação de 
cada funcionalidade do sistema.

No Capítulo \ref{ch:IM}, a implementação é conduzida por meio da execução das quatro primeiras etapas do processo. Essa fase tem início com a definição das histórias 
de usuário — que representam as funcionalidades do sistema — conforme descrito na Seção \ref{sbs:etapa1}. As histórias orientam todas as etapas subsequentes, desde a 
elaboração dos testes de aceitação até a modelagem do sistema.

Para cada história de usuário, o mapeamento de exemplos, descrito na Seção \ref{sbs:etapa2}, será empregado para orientar a definição dos detalhes relacionados 
ao comportamento esperado. Em seguida, na Seção \ref{sbs:etapa3}, o sistema será modelado como uma abstração que explicita as interações com o usuário, ao mesmo 
tempo em que oculta os aspectos técnicos de sua implementação. A partir das histórias de usuário e do modelo do sistema, será elaborada, na Seção \ref{sbs:etapa4}, 
uma série de cenários que representam tanto a especificação do comportamento esperado quanto os testes de aceitação correspondentes.  

No Capítulo \ref{ch:RD}, serão executados os testes de validação definidos pelos cenários das histórias de usuário. Esse processo assegura que a modelagem do sistema 
seja orientada pela execução dos testes e que o comportamento especificado seja atendido. Ela também possibilita a identificação e a correção de falhas detectadas 
durante a execução dos testes.

A aprovação das histórias de usuário será realizada a partir da verificação de que:

\begin{itemize}
    \item Os testes foram executados e confirmaram a aprovação de todos os cenários relacionados à história;
    \item A execução dos cenários é repetida ao longo do desenvolvimento das demais histórias, de modo a assegurar que o comportamento previamente validado não seja comprometido.
\end{itemize}

Por fim, no Capítulo \ref{ch:CC}, será apresentada uma síntese dos principais aspectos observados durante a aplicação da metodologia proposta. Adicionalmente, serão 
discutidas algumas limitações do escopo deste trabalho, acompanhadas dos passos futuros necessários para a continuidade do desenvolvimento e geração 
do produto físico.
