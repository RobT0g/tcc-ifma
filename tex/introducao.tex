%--------------------------------------------------------------------------------------
% Este arquivo contém a sua introdução, objetivos e organização do trabalho
%--------------------------------------------------------------------------------------
\chapter{Introdução}

A indústria automotiva tem experimentado um crescimento constante, acompanhado por um aumento significativo na modernização dos veículos e pela intensa competição entre 
montadoras para entregar produtos cada vez mais robustos e de alta qualidade. Esse cenário traz consigo um aumento na complexidade do desenvolvimento, demandando soluções 
mais eficientes e seguras para sistemas cada vez mais sofisticados e que sejam capazes de atender às necessidades específicas de cada cliente.

A modernização do setor automotivo tem conduzido ao próximo grande avanço: os veículos definidos por software (\textit{Software-Defined Vehicles}) 
\cite{softwareVehicles2025}. Essa abordagem possibilita a customização contínua e a evolução do produto mesmo após a aquisição, permitindo a entrega de novas 
funcionalidades ao veículo por meio de atualizações de software feitas remotamente \textit{Over-the-Air (OTA)}, suportadas pelo hardware já embarcado.

Diante desse contexto, a Engenharia Automotiva enfrenta o desafio de projetar e desenvolver produtos com um nível altíssimo de qualidade, ao mesmo tempo em que busca reduzir 
erros e retrabalho. A aplicação de metodologias ágeis, como o \textit{Behavior Driven Development (BDD)} \cite{north2006bdd}, oferece uma abordagem promissora, pois foca na 
definição do comportamento esperado do sistema a partir da perspectiva do usuário. Essa abordagem permite uma colaboração mais efetiva entre os membros de uma equipe ágil 
\cite{atlassianAgileTeams}, facilita a descoberta de funcionalidades e possibilita que o desenvolvimento seja guiado por testes de aceitação automatizados, promovendo 
entregas de valor de forma mais rápida e com maior confiabilidade.

\section{\textbf{Justificativa}}

O problema central deste trabalho é: “como é possível aplicar o processo de BDD para facilitar a definição do sistema e guiar seu desenvolvimento no contexto da 
Engenharia de Sistemas Automotivos?”. A solução proposta consiste na adaptação do processo, integrando as técnicas já consolidadas do BDD com metodologias 
amplamente utilizadas na indústria automotiva, de forma a lidar com as complexidades inerentes ao desenvolvimento de sistemas veiculares.

O BDD incorpora diversas premissas que surgem das lições aprendidas na aplicação de outras metodologias ágeis, definindo o desenvolvimento do produto a partir do 
comportamento que gera valor ao usuário. Além disso, ele busca promover a colaboração entre as diferentes etapas do desenvolvimento, garantindo que toda a equipe 
compreenda o produto e consiga assumir a perspectiva do usuário.

Por outro lado, a Engenharia Automotiva apresenta uma complexidade significativa, o que torna desafiadora a entrega de produtos de alta qualidade. A definição de 
um sistema veicular envolve muito mais do que software ou eletrônica \cite{bosch2022handbook}, sendo necessário integrar componentes mecânicos, considerar falhas 
potenciais, garantir a rastreabilidade de requisitos e contemplar casos de uso extremos.

O principal desafio deste trabalho está em identificar técnicas capazes de capturar essas complexidades e torná-las compatíveis com as premissas do BDD.

\section{\textbf{Objetivos}}

\subsection{Objetivo Geral}
%Para responder à problemática levantada, o presente estudo tem como objetivo geral 
Aplicar o levantamento e definição de funcionalidades na análise de um sistema de travamento de portas veicular, seguido do desenvolvimento guiado por testes de 
aceitação automatizados, demonstrando como essas práticas podem contribuir para o desenvolvimento de software automotivo de forma mais organizada e eficiente.

\subsection{Objetivos Específicos}
\begin{itemize}
    \item Realizar o levantamento e definição de funcionalidades considerando possíveis limitações decorrentes da dependência de hardware e de sistemas mecânicos;
	\item Analisar as complexidades dos sistemas automotivos de maneira compatível com o processo de BDD;
    \item Empregar técnicas de desenvolvimento como padrão de escrita de software AUTOSAR \cite{autosarClassic} e Design Orientado por Modelos (\textit{Model-Based Desgin - MBD}) \cite{mathworksMBD2024}, que são amplamente utilizadas na indústria automotiva;
    \item Definir um escopo de desenvolvimento que possibilite a realização do design do sistema forma clara e compreensível;
    \item Empregar testes de aceitação automatizados para guiar o desenvolvimento do sistema, consistente com o design definido;
    \item Realizar a implementação do sistema de maneira iterativa, possiblitando à descoberta de melhorias surgidas durante a aplicação do processo;
    \item Aplicar a metodologia desenvolvida na produção de um sistema de travamento de portas veicular simulado.
\end{itemize}

% \section{Metodologia}

% A metodologia utilizada neste estudo baseia-se na revisão da bibliografia disponível, com o objetivo de esclarecer o processo de BDD e investigar como ele pode 
% ser adaptado para incorporar práticas consolidadas na Engenharia Automotiva. Para tanto, serão aplicadas técnicas como o mapeamento de exemplos 
% \cite{Lawrence2019cucumber, cucumberExampleMapping}
% , voltado ao levantamento e definição de funcionalidades, e testes de aceitação automatizados 
% \cite{studyBDD}
% , que servirão para guiar o desenvolvimento do sistema.

% A implementação do sistema simulado também fará uso de técnicas típicas da Engenharia Automotiva, incluindo o Model Based Design
% \cite{Vincentelli2001, mathworksMBD2024} e a geração automática de código segundo o padrão de software AUTOSAR 
% \cite{autosarClassic}. Conceitos adicionais do domínio automotivo serão considerados para 
% assegurar a adequada abstração nas diferentes etapas do desenvolvimento.

% Espera-se que os resultados obtidos evidenciam os benefícios da integração do BDD no contexto automotivo, destacando sua contribuição para a qualidade e 
% confiabilidade do sistema. A validação será realizada por meio da execução dos testes de aceitação e da análise do uso do sistema final sob a perspectiva do usuário.

\section{Organização do trabalho}
Este trabalho está estruturado da seguinte forma: o capítulo 2 apresenta a fundamentação teórica sobre BDD e testes automatizados, 
o capítulo 3 descreve a metodologia aplicada, o capítulo 4 detalha a implementação e os resultados obtidos, e o capítulo 5 apresenta as conclusões e considerações finais.
