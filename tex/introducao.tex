%--------------------------------------------------------------------------------------
% Este arquivo contém a sua introdução, objetivos e organização do trabalho
%--------------------------------------------------------------------------------------
\chapter{Introdução}

A indústria automotiva tem experimentado um crescimento constante, acompanhado por um aumento significativo na modernização dos veículos e pela intensa competição entre 
montadoras para entregar produtos cada vez mais robustos e de alta qualidade. Esse cenário traz consigo um aumento na complexidade do desenvolvimento, demandando soluções 
mais eficientes e seguras para sistemas cada vez mais sofisticados e que sejam capazes de atender às necessidades específicas de cada cliente.

Diante desse contexto, a Engenharia Automotiva enfrenta o desafio de projetar e desenvolver produtos com um nível altíssimo de qualidade, ao mesmo tempo em que busca reduzir 
erros e retrabalho. A aplicação de metodologias ágeis, como o Behavior Driven Development (BDD) \cite{north2006bdd}, oferece uma abordagem promissora, pois foca na definição 
do comportamento esperado do sistema a partir da perspectiva do usuário. Essa abordagem permite uma colaboração mais efetiva entre os membros de uma equipe ágil 
\cite{atlassianAgileTeams}, facilita a descoberta de funcionalidades e possibilita que o desenvolvimento seja guiado por testes de aceitação automatizados, promovendo entregas de valor de forma 
mais rápida e com maior confiabilidade.

\section{Problemática}

O problema central deste trabalho é: “como é possível aplicar o processo de BDD para facilitar a definição do sistema e guiar seu desenvolvimento no contexto da 
Engenharia de Sistemas Automotivos?”.
O BDD incorpora diversas premissas que surgem das lições aprendidas na aplicação de outras metodologias ágeis, definindo o desenvolvimento do produto a partir do 
comportamento que gera valor ao usuário. Além disso, ele busca promover a colaboração entre as diferentes etapas do desenvolvimento, garantindo que toda a equipe 
compreenda o produto e consiga assumir a perspectiva do usuário.
Por outro lado, a Engenharia Automotiva apresenta uma complexidade significativa, o que torna desafiadora a entrega de produtos de alta qualidade. A definição de 
um sistema veicular envolve muito mais do que software ou eletrônica \cite{bosch2022handbook}, sendo necessário integrar componentes mecânicos, considerar falhas 
potenciais e contemplar casos de uso extremos.
O principal desafio deste trabalho está em identificar técnicas capazes de capturar essas complexidades e torná-las compatíveis com as premissas do BDD.

\section{Objetivos gerais e específicos}

Para responder à problemática levantada, o presente estudo tem como objetivo geral aplicar o levantamento e definição de funcionalidades na análise de um sistema de 
travamento de portas veicular, seguido do desenvolvimento guiado por testes de aceitação automatizados, demonstrando como essas práticas podem contribuir para o 
desenvolvimento de software automotivo de forma mais organizada e eficiente.
Para isso, os objetivos específicos serão focados em:
\begin{itemize}
	\item Abordar as complexidades dos sistemas automotivos de maneira compatível com o processo de BDD;
    \item Empregar técnicas de desenvolvimento que são amplamente utilizadas na indústria automotiva;
    \item Realizar o levantamento e definição de funcionalidades considerando possíveis limitações decorrentes da dependência de hardware e de sistemas mecânicos;
    \item Definir um escopo de desenvolvimento que possibilite a realização do design do sistema forma clara e compreensível;
    \item Empregar testes de aceitação automatizados para possibilitar o desenvolvimento do sistema de maneira iterativa, consistente com o design definido;
    \item Possibilitar a adaptação às descobertas surgidas durante a aplicação do processo, garantindo a melhoria contínua do design do sistema e do valor entregue ao usuário.
\end{itemize}

\section{Metodologia}

A metodologia utilizada neste estudo baseia-se na revisão da bibliografia disponível, com o objetivo de esclarecer o processo de BDD e investigar como ele pode 
ser adaptado para incorporar práticas consolidadas na Engenharia Automotiva. Para tanto, serão aplicadas técnicas como o mapeamento de exemplos 
\cite{gosling2014cucumber, cucumberExampleMapping}
, voltado ao levantamento e definição de funcionalidades, e testes de aceitação automatizados 
\cite{studyBDD}
, que servirão para guiar o desenvolvimento do sistema.
A implementação do sistema simulado também fará uso de técnicas típicas da Engenharia Automotiva, incluindo o Model Based Design \cite{sangiovanni2001platform, mathworks2021mbd}
e a geração automática de código segundo o padrão de software AUTOSAR 
\cite{autosarClassic}. Conceitos adicionais do domínio automotivo serão considerados para 
assegurar a adequada abstração nas diferentes etapas do desenvolvimento.
Espera-se que os resultados obtidos evidenciam os benefícios da integração do BDD no contexto automotivo, destacando sua contribuição para a qualidade e 
confiabilidade do sistema. A validação será realizada por meio da execução dos testes de aceitação e da análise do uso do sistema final sob a perspectiva do usuário.

\section{Organização do trabalho}
Este trabalho está estruturado da seguinte forma: o capítulo 2 apresenta a fundamentação teórica sobre BDD e testes automatizados, 
o capítulo 3 descreve a metodologia aplicada, o capítulo 4 detalha a implementação e os resultados obtidos, e o capítulo 5 apresenta as conclusões e considerações finais.
