\chapter{Resultados} \label{ch:RD}



% \section{Seção de exemplo 1 - Códigos} \label{sec:resex1}

% \subsection{Subseção de exemplo 1 - Inserindo trechos de códigos}
 
% O nosso querido Leonardo Cavalcante providenciou um comando que deixa nossos trechos de códigos bonitinhos e gera um elemento pré-textual de Lista de Códigos. 

% Os códigos são adicionados através do comando seguinte:

% \textbackslash sourcecode\{ Descrição \}\{Label\}\{Linguagem\}\{Arquivo com extensão\}

% Um exemplo pode ser visto no código \ref{cmd:cron} abaixo.

% %\sourcecode{Configuração do intervalo de execução no Script Agendador}{cron}{javascript}{cron.js}


% \section{Seção de exemplo 2 - Listas} \label{sec:resex2}

% \subsection{Subseção de exemplo 2 - Lista de itens} 

% Existem alguns tipos de listas no Latex, iremos exemplificar a lista sem numeração (seção \ref{subsubsec:itemize}), a lista enumerada (seção \ref{subsubsec:enumerate}) e a lista mista (seção \ref{subsubsec:mista}). As listas podem ser encadeadas de diversas maneiras,
% de acordo com a necessidade do autor.

% \subsubsection{Subsubseção de exemplo 1 - Lista sem numeração} \label{subsubsec:itemize}

% Este é um exemplo de lista sem numeração.

% \begin{itemize}
% 	\item \textbf{Cadastrar usuário}

% 		\begin{itemize}
%     		\item Atores
% 		    	\begin{itemize}
%     		    	\item Usuário
% 		    	\end{itemize}

% 	    	\item Fluxo de eventos primário
% 			    \begin{itemize}
% 	    		    \item o usuário deve se cadastrar informando seu nome, \textit{e-mail} e senha;
% 		        	\item a API armazena os dados do usuário;
% 		    	    \item o usuário é liberado para realizar o \textit{login}.
% 			    \end{itemize}

%     		\item Fluxo alternativo
% 			    \begin{itemize}
% 		    	   \item o usuário desiste de se cadastrar e cancela o caso de uso clicando no botão voltar.
% 	    		\end{itemize}

% 		\end{itemize}
	
% \end{itemize}

% \subsubsection{Subsubseção de exemplo 2 - Lista enumerada} \label{subsubsec:enumerate}

% Este é um exemplo de lista enumerada.

% \begin{enumerate}
% 	\item O Usuário deseja ver o histórico das variáveis climáticas, então através da interface de usuário escolhe o período ao qual o histórico se refere;
% 	\item A aplicação solicita à API através de uma requisição HTTP contendo o momento de início e o momento do fim do período em seus parâmetros;     			\item A API recebe a solicitação e se comunica com a base de dados, então requere as informações quem possuem a data de leitura no intervalo escolhido;
% 	\item A base de dados retorna os dados em formato Json para a API;
% 	\item A API responde à requisição retornando os dados, também em formato Json, para a aplicação cliente;
% 	\item A aplicação cliente renderiza os gráficos utilizando o conjunto de dados obtidos.
% \end{enumerate}

% \subsubsection{Subsubseção de exemplo 3 - Lista mista} \label{subsubsec:mista}

% Este é um exemplo de lista mista.

% \begin{itemize}
% 	\item \textbf{Cadastrar usuário}

% 		\begin{itemize}
%     		\item Atores
% 		    	\begin{itemize}
%     		    	\item Usuário
% 		    	\end{itemize}

% 	    	\item Fluxo de eventos primário
% 			    \begin{enumerate}
% 	    		    \item o usuário deve se cadastrar informando seu nome, \textit{e-mail} e senha;
% 		        	\item a API armazena os dados do usuário;
% 		    	    \item o usuário é liberado para realizar o \textit{login}.
% 			    \end{enumerate}

%     		\item Fluxo alternativo
% 			    \begin{itemize}
% 		    	   \item o usuário desiste de se cadastrar e cancela o caso de uso clicando no botão voltar.
% 	    		\end{itemize}

% 		\end{itemize}

% 	\item \textbf{Visualizar dados atuais}

% 		\begin{itemize}
% 		    \item Atores
% 	    		\begin{itemize}
% 		    	    \item Usuário
% 			    \end{itemize}
    
% 	    	\item Pré-condições
% 			    \begin{itemize}
% 		     	   \item o usuário deve estar autenticado
% 			    \end{itemize}

% 	    	\item Fluxo de eventos primário
% 			    \begin{enumerate}
% 		    	    \item o usuário deve efetuar o \textit{login} informando o \textit{e-mail} e a senha;
% 	    		    \item caso o usuário não seja autenticado, o sistema informa a respeito de credenciais inválidas e encerra o caso de uso;
% 		    	    \item a API autentica o usuário;
%     			    \item o usuário é liberado para visualizar os dados atuais dos sensores da estação;
% 		        	\item após a visualização o usuário pode finalizar o caso de uso ou efetuar uma nova consulta se desejar.
% 			    \end{enumerate}

%     		\item Fluxo alternativo
% 			    \begin{itemize}
%     			   \item o usuário desiste de visualizar os dados atuais e cancela o caso de uso clicando no botão voltar.
% 			    \end{itemize}

% 		\end{itemize}

% 	\item \textbf{Visualizar histórico}

% 		\begin{itemize}
% 		    \item Atores
% 	    		\begin{itemize}
% 		    	    \item Usuário
% 	    		\end{itemize}

% 	    	\item Pré-condições
%     			\begin{itemize}
% 			        \item o usuário deve estar autenticado
% 			    \end{itemize}

% 		    \item Fluxo de eventos primário
% 			    \begin{enumerate}
% 			        \item o usuário deve efetuar o \textit{login} informando o \textit{e-mail} e a senha;
% 			        \item caso o usuário não seja autenticado, o sistema informa a respeito de credenciais inválidas e encerra o caso de uso;
% 			        \item a API autentica o usuário;
% 			        \item o usuário é liberado para escolher qual período cujo histórico será exibido;
% 			        \item o usuário seleciona as variáveis a serem exibidas no gráficos de linhas;
% 			        \item após a visualização do histórico o usuário pode finalizar o caso de uso se desejar.
% 			    \end{enumerate}

% 		    \item Fluxo alternativo
% 			    \begin{enumerate}
% 			        \item após a escolha do período de exibição do histórico o usuário pode voltar para a tela anterior e escolher um novo período;
% 			        \item o histórico é exibido para o usuário;
% 			        \item após a visualização do histórico o usuário pode finalizar o caso de uso ou efetuar uma nova consulta se desejar.
% 			    \end{enumerate}

% 		    \item Fluxo alternativo
% 			    \begin{enumerate}
% 			        \item o usuário desiste de visualizar o histórico e cancela o caso de uso clicando no botão voltar.
% 			    \end{enumerate}
% 		\end{itemize}
% \end{itemize}

\section{Desenvolvimento iterativo}

Após a finalização da criação dos cenários Gherkin, a implementação prossegue com o desenvolvimento do modelo e das definições de passos. Inicialmente, o modelo é 
criado dentro da estrutura do projeto, em uma nova pasta denominada \textit{model}. O conteúdo desta pasta inclui:

\begin{itemize}
	\item \textbf{feature\_model.slx} - modelo caixa preta, utilizado nos testes;
	\item \textbf{main.slx} - modelo de caixa branca, que contém a lógica do sistema e é referenciado como um bloco subsystem dentro do modelo feature\_model.slx;
	\item \textbf{write\_to\_model.m} - função MATLAB responsável por operações de escrita no modelo;
	\item \textbf{read\_from\_model.m} - função MATLAB responsável por operações de leitura no modelo;
	\item \textbf{initialize\_model.m} - função matlab que inicia o modelo e torna a API da aplicação disponível para as funções das definições de passo.
\end{itemize}

O modelo \textbf{feature\_model.slx} segue a estrutura do diagrama de caixa preta, contendo apenas as entradas e saídas do sistema, sem detalhar sua lógica interna. Para 
isso, são utilizados blocos de constantes como entradas e \textit{displays} como saídas.

Durante a execução dos testes, as definições de passos simulam o comportamento do sistema ao interagir com suas entradas, modificando os valores das constantes por 
meio da função \textbf{write\_to\_model}. Para validar a resposta do sistema, os valores dos displays de saída, determinados pela lógica interna do modelo, são obtidos utilizando 
a função \textbf{read\_from\_model}.

\begin{figure}[H]
\centering
\includegraphics[height=7cm]{figuras/feature\_model.png}
\caption{Modelo \textbf{feature\_model.slx} que aplica o conceito de caixa preta.}
%\label{fig:casos-uso}
\end{figure}

O modelo apresentado na imagem contém entradas auxiliares que ainda não estão contempladas no diagrama de caixa preta. Essas entradas foram adicionadas para tornar certos 
comportamentos do sistema testáveis durante o desenvolvimento do modelo e serão detalhadas ao longo desta seção.

O segundo modelo, \textbf{main.slx}, é referenciado no bloco central do modelo anterior, denominado controller. Ele é responsável pela implementação da lógica interna 
do sistema em desenvolvimento, utilizando as entradas definidas como blocos \textit{Inport} e as saídas como \textit{Outport}, conectadas às constantes e \textit{displays} 
do \textbf{feature\_model.slx}.

Em seguida, o desenvolvimento do environment.py inclui funções em Python que tornam as operações de escrita e leitura do modelo acessíveis às definições de passos. 
Para isso, são criadas diversas funções utilitárias, permitindo realizar ações como:

\begin{itemize}
	\item Iniciar a simulação;
	\item Parar a simulação;
	\item Executar operações de escrita no modelo;
	\item Executar operações de leitura no modelo.
\end{itemize}

Essas funções utilitárias ficam disponíveis por meio do objeto context, o qual pode ser acessado diretamente na lógica das funções das definições de passos. Com isso, 
cada passo consegue interagir com o modelo ao executar as operações especificadas no arquivo \textbf{environment.py}.

Para garantir uma execução iterativa dos testes, cada história é validada individualmente em um primeiro momento. Após a sua aprovação, todas as histórias anteriores 
são executadas novamente, assegurando que a implementação de novas funcionalidades não comprometa o comportamento já estabelecido.

A execução dos testes é realizada após a definição da lógica interna das funções de cada passo. Os resultados são então documentados em um relatório automático, gerado 
em formato \textbf{.xml}, que descreve detalhadamente todas as falhas encontradas, conforme o seguinte padrão:

\begin{figure}[H]
\centering
\includegraphics[height=3cm]{figuras/teste\_feature.png}
\caption{Relatório de testes - arquivo \textbf{.feature}.}
%\label{fig:casos-uso}
\end{figure}

O relatório também apresenta os resultados da execução de passos específicos, organizados da seguinte forma:

\begin{figure}[H]
\centering
\includegraphics[height=5cm]{figuras/teste\_cenario.png}
\caption{Relatório de testes - por cenário.}
%\label{fig:casos-uso}
\end{figure}

Para facilitar a análise de falhas, as funções implementadas em \textbf{environment.py} registram no relatório todas as interações realizadas com o modelo. Isso permite validar 
manualmente se a execução dos testes ocorreu conforme o esperado.


