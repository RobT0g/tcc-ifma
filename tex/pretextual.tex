%--------------------------------------------------------------------------------
% Constrói a capa com base na seção de identificação do main.tex
%--------------------------------------------------------------------------------
\begin{capa}
    \setlength{\belowcaptionskip}{0pt}
    \setlength{\abovecaptionskip}{0pt}
    \setlength{\intextsep}{-18pt}
        \begin{figure}[h]
    		\begin{center}
    		    \includegraphics[scale=.5]{img/ifma_novo.jpg}
    		\end{center}
    	\end{figure}

        %\includegraphics[scale=0.6]{img/univasf.jpg}
        \center
    	{\ABNTEXchapterfont\large\imprimirinstituicao}

    	\vspace*{2cm}
    	    {\imprimirautor}
    	\vspace*{2cm}
        \begin{center}
    		\ABNTEXchapterfont\bfseries\large\imprimirtitulo
        \end{center}
    	\vfill

    	\ABNTEXchapterfont\bfseries\large\imprimirlocal\\
    	\the\year

    	\vspace*{1cm}
\end{capa}
%--------------------------------------------------------------------------------
% Constrói a folha de rosto com base na seção de identificação do main.tex
%--------------------------------------------------------------------------------
\begin{folhaderosto}
    \center
    	{\ABNTEXchapterfont\large\imprimirinstituicao}

		\vspace*{2cm}
    	    {\imprimirautor}
    	\vspace*{2cm}
		\vspace*{\fill}

		{\ABNTEXchapterfont\bfseries\large\imprimirtitulo}
		\vspace*{\fill}

		{\hspace{.45\textwidth}
		\begin{minipage}{.5\textwidth}
			\SingleSpacing
			\imprimirpreambulo \\ \\

			{\imprimirorientadorRotulo~\imprimirorientador\par}
			{\imprimircoorientadorRotulo~\imprimircoorientador\par}

		\end{minipage}%
		\vspace*{\fill}}%
		\vspace*{\fill}
			\ABNTEXchapterfont\bfseries\large\imprimirlocal\\
			\the\year
		\vspace*{1cm}
\end{folhaderosto}

%--------------------------------------------------------------------------------
% Constrói a ficha catalográfia com base na seção de identificação do main.tex
% Está comentado porque no final das contas a biblioteca do seu campus que gera a
% numeração, você pode adicionar os numeros aqui, ou anexar o pdf gerado por eles
% ao documento.
%--------------------------------------------------------------------------------
%\begin{fichacatalografica}
%	\vspace*{\fill}					% Posição vertical
%	\hrule							% Linha horizontal
%	\begin{center}					% Minipage Centralizado
%	\begin{minipage}[c]{12.5cm}		% Largura
%
%	\imprimirautor
%
%	\hspace{0.5cm} \imprimirtitulo  / \imprimirautor. --
%	\imprimirlocal, \the\year-
%
%	\hspace{0.5cm} xx p. : il. (algumas color.) ; 30 cm.\\
%
%	\hspace{0.5cm} \imprimirorientadorRotulo~\imprimirorientador\\
%
%	\hspace{0.5cm}
%	\parbox[t]{\textwidth}{\imprimirtipotrabalho~--~\imprimirinstituicao,
%	\the\year.}\\
%
%	\hspace{0.5cm}
%		1. Palavra-chave1.
%		2. Palavra-chave2.
%		I. Orientador.
%		II. Universidade xxx.
%		III. Faculdade de xxx.
%		IV. Título\\
%
%	\hspace{8.75cm} CDU 02:141:005.7\\
%
%	\end{minipage}
%	\end{center}
%	\hrule
%\end{fichacatalografica}

%--------------------------------------------------------------------------------
% Anexando a ficha catalogáfica e a folha de aprovação
%--------------------------------------------------------------------------------
\includepdf[pages=-]{anexos/ficha.pdf}

\includepdf[pages=-]{anexos/aprovacao.pdf}

\setlength{\ABNTEXsignwidth}{12cm}

%--------------------------------------------------------------------------------
% Está comentado pelo mesmo motivo da ficha catalográfica
%--------------------------------------------------------------------------------
%\begin{folhadeaprovacao}
%	\begin{center}
%	    {\ABNTEXchapterfont\bfseries\large\imprimirinstituicao}
%	    \vspace*{\fill}
%
%	    {\ABNTEXchapterfont\bfseries\large FOLHA DE APROVAÇÃO}
%	    \vspace*{\fill}
%
%	    {\ABNTEXchapterfont\bfseries\large\imprimirautor}
%
%	    \vspace*{\fill}\vspace*{\fill}
%	    {\ABNTEXchapterfont\bfseries\large\imprimirtitulo}
%	    \vspace*{\fill}
%
%	    {\hspace{.45\textwidth}
%		\begin{minipage}{.5\textwidth}
%			\SingleSpacing
%			\ABNTEXchapterfont\imprimirpreambulo \\ \\
%
%			{\ABNTEXchapterfont\imprimirorientadorRotulo~\imprimirorientador\par}
%			{\ABNTEXchapterfont\imprimircoorientadorRotulo~\imprimircoorientador\par}
%
%		\end{minipage}%
%	    \vspace*{\fill}}
%	\end{center}
%
%	\vspace*{\fill}
%
%	\begin{center}
%			 \ABNTEXchapterfont\large Aprovado em: \_\_\_\_ de \_\_\_\_ de 2017
%	\end{center}

%	\vspace*{\fill}

%	\begin{center}
%			 \ABNTEXchapterfont\bfseries\large Banca Examinadora
%	\end{center}
%
%   \ABNTEXchapterfont\assinatura{Fábio Nelson de Sousa Pereira, Mestre, Universidade Federal do Vale do São Francisco}
%	\ABNTEXchapterfont\assinatura{Jorge Luis Cavalcanti Ramos, Doutor, Universidade Federal do vale do São Francisco}
%  \ABNTEXchapterfont\assinatura{Ricardo Argenton Ramos, Doutor, Universidade Federal do Vale do São Francisco}
%	 \vspace*{\fill}


%\end{folhadeaprovacao}

% %--------------------------------------------------------------------------------
% % Insere a epígrafe
% %--------------------------------------------------------------------------------
% \newpage
% \vspace*{\fill}
% \begin{flushright}
% 		\textit{Lorem Ipsum...}
% \end{flushright}

%--------------------------------------------------------------------------------
% Seção de agradecimentos
%--------------------------------------------------------------------------------
\begin{agradecimentos}

%\vspace*{\fill}
		Agradeço o IFMA e ao curso de Engenharia da Computação, pela formação acadêmica e pelas oportunidades oferecidas ao longo da graduação.
		Aos professores Msc. Emanuel Cleyton Macedo Lemos e Dr. Aristoteles de Almeida Lacerda Neto, pela orientação e por toda a ajuda durante a minha trajetória de aprendizado.
		Aos meus pais, minha irmã e toda a minha família pelo suporte e pela ajuda nos momentos difíceis.
		\vspace{\baselineskip}

\end{agradecimentos}

% %--------------------------------------------------------------------------------
% % Insere a segunda epígrafe
% %--------------------------------------------------------------------------------
% \begin{epigrafe}
%     \vspace*{\fill}
% 	\begin{flushright}
% 		Se pude enxergar a tão grande distância, foi subindo nos ombros de gigantes.\\
% 		 \vspace{\baselineskip}
% 		\textbf{Isaac Newton}\\
% 		\textbf{Carta à Robert Hooke, 1676}
% 	\end{flushright}
% \end{epigrafe}



%--------------------------------------------------------------------------------
% Seção de resumos
%--------------------------------------------------------------------------------
% resumo em português
\setlength{\absparsep}{18pt} % ajusta o espaçamento dos parágrafos do resumo
\begin{resumo}

	O presente trabalho surgiu a partir das dificuldades identificadas nos processos de desenvolvimento de sistemas veiculares, que apresentam elevada complexidade 
	em relação à definição, implementação e validação de produtos, desafios constantemente enfrentados pelas grandes montadoras. Com o objetivo de propor uma 
	abordagem capaz de mitigar esses problemas, este estudo adapta a metodologia de Desenvolvimento Orientado por Comportamento (Behavior-Driven Development — BDD) 
	ao contexto da Engenharia Automotiva, empregando o mapeamento de exemplos e a linguagem Gherkin para a definição de funcionalidades, geração de testes de 
	aceitação automatizados e guiar a implementação do sistema. O processo inclui adaptações específicas, integrando práticas já consolidadas na indústria para
	garantir a sua compatibilidade com o desenvolvimento de software veicular. Entre os principais benefícios observados, destacam-se a melhor qualidade na definição 
	de funcionalidades, maior cobertura de testes e a produção de documentação viva, capaz de validar continuamente o comportamento do sistema frente aos resultados 
	esperados. A metodologia proposta foi aplicada na construção de um protótipo que simula um sistema automotivo de travamento de portas, atendendo às especificações de 
	comportamento definidas e priorizando o valor entregue ao usuário.
	\vspace{\onelineskip}
    \noindent

	\textbf{Palavras-chave}: \textit{BDD. Gherkin. teste-automatizado. engenharia-automotiva. desenvolvimento-ágil.}

\end{resumo}

%---------------------------------------------------------------------------------
% resumo em inglês
\begin{resumo}[Abstract]
\begin{otherlanguage*}{english}

	This work originated from the difficulties identified in the development processes of automotive systems, which present a high level of complexity 
	regarding the definition, implementation, and validation of products, challenges frequently faced by big manufacturers. Aiming to propose an approach 
	capable of mitigating these issues, this study adapts the Behavior-Driven Development (BDD) methodology to the context of Automotive Engineering, 
	employing example mapping and the Gherkin language for defining functionalities, generating automated acceptance tests and guiding the system implementation. 
	The	process includes specific adaptations, integrating practices already consolidated in the industry to ensure compatibility with automotive software development. 
	Among the main benefits observed are broader test coverage and the creation of living documentation, capable of continuously validating the system's behavior 
	against the expected results. The proposed methodology was applied to the development of a prototype simulating an automotive door-locking system, meeting the 
	defined behavioral specifications and prioritizing the value delivered to the user.
	\vspace{\onelineskip}
	\noindent

	\textbf{Key-words}: \textit{BDD. Gherkin. automated-tests. automotive-engineering. agile-development.}

\end{otherlanguage*}
\end{resumo}


%---------------------------------------------------------------------------------
% Insere lista de ilustrações
%---------------------------------------------------------------------------------
\begin{KeepFromToc} % Este comando evita que todas as seções dentro dele de apareçam no sumário
\pdfbookmark[0]{\listfigurename}{lof}
\listoffigures



\cleardoublepage


% %---------------------------------------------------------------------------------
% % Insere lista de tabelas
% %---------------------------------------------------------------------------------
% \pdfbookmark[0]{\listtablename}{lot}
% \listoftables
% \cleardoublepage

%---------------------------------------------------------------------------------
% Insere lista de quadros
%---------------------------------------------------------------------------------
\pdfbookmark[0]{\listofquadrosname}{loq}
\listofquadros
\cleardoublepage

% %---------------------------------------------------------------------------------
% % Ajusta lista de código - alterar de figures para códigos - by @Gabrielr2508
% %---------------------------------------------------------------------------------
% \makeatletter
% \let\l@listing\l@figure
% \def\newfloat@listoflisting@hook{\let\figurename\listingname}
% \makeatother

% %---------------------------------------------------------------------------------
% % Insere lista de códigos - by @leolleocomp
% %---------------------------------------------------------------------------------
% \listoflistings

\end{KeepFromToc}

% %---------------------------------------------------------------------------------
% % Insere lista de abreviaturas e siglas
% %---------------------------------------------------------------------------------
% \begin{siglas}
% 	\item[LI]       Lorem Ipsum
%     \item[LII]		Lorem Ipsum Ipsum

% \end{siglas}

%---------------------------------------------------------------------------------
% Insere o sumario
%---------------------------------------------------------------------------------
\pdfbookmark[0]{\contentsname}{toc}
\tableofcontents*
\cleardoublepage


