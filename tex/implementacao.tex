\chapter{Implementação} \label{ch:IM}

Neste capítulo será demonstrado em detalhes o procedimento da aplicação de desenvolvimento de produto como anteriormente apresentado, 
com 9 etapas que expandem o modelo de \cite{Lawrence2019cucumber} para adaptá-lo ao produto em questão.

\section{Etapa 1: Descrição das histórias de usuário}
A etapa inicial do Desenvolvimento Orientado por Comportamento (BDD) consiste na definição das funcionalidades a serem implementadas por meio de histórias de usuário. 
Nessa fase, a perspectiva do usuário — ou da pessoa que se beneficia da funcionalidade — deve ser o foco principal. Essa abordagem é um dos pilares fundamentais do BDD, 
pois garante que o produto desenvolvido entregue valor real ao usuário final.

Portanto, no contexto deste trabalho que trata do sistema de travamento de portas veicular, serão consideradas implementações práticas amplamente adotadas pela 
indústria automotiva como base para as histórias de usuário criadas.

O travamento e destravamento remoto das portas, acionado por meio da chave com controle remoto ou até mesmo via celular, permite que todas as portas do veículo sejam 
travadas ou destravadas à distância, sem a necessidade de inserção da chave na fechadura \cite{bosch2022handbook}. Esse caso de uso mais simples é vastamente presente 
em veículos de diversas montadoras e pode ser capturado por meio de duas histórias de usuário:

\begin{itemize}
    \item Como dono de um veículo automotivo, quero travar todas as portas do meu carro utilizando o botão de travamento, para que eu possa mantê-lo seguro contra furtos e impedir o acesso de pessoas não autorizadas;
    \item Como dono de um veículo automotivo, quero destravar todas as portas do meu carro utilizando o botão de destravamento, para que eu possa acessá-lo.
\end{itemize}

Além disso, outra funcionalidade que é também vastamente presente é a indicação da confirmação do travamento ou destravamento das portas, que normalmente utiliza um 
sinal visual com os faróis do carro, assim como descrito por \cite{vwLocking}. A seguinte história de usuário será implementada para capturar essa funcionalidade:

\begin{itemize}
    \item Como dono de um veículo automotivo, quero receber uma indicação clara sempre que o travamento ou destravamento das portas ocorrer, para que eu tenha certeza do estado de segurança do meu carro.
\end{itemize}

Em alguns veículos como os da Volkswagen \cite{vwLocking}, há uma funcionalidade que permite destravar as portas sem a necessidade de usar a chave, desde que ela 
esteja próxima ao veículo. O sistema \texttt{Keyless Access} possibilita que, ao interagir com as maçanetas das portas, o motorista possa destravar ou travar o carro sem 
precisar pressionar o botão no controle.

Esse recurso é útil em situações cotidianas, como quando o motorista deseja entrar no carro enquanto a chave está, por exemplo, no bolso. Embora o processo 
tradicional exija que o usuário retire a chave do bolso, pressione o botão de destravamento e, em seguida, guarde-a novamente, o uso do \texttt{Keyless Access} simplifica 
essa sequência. A diferença é pequena — apenas alguns segundos de tempo —, porém a implementação de uma funcionalidade que elimina etapas desnecessárias torna a 
experiência mais prática, especialmente considerando que esse mesmo uso será repetido diversas vezes ao longo do dia.

O manual do veículo descreve funcionalidades adicionais, como o travamento das portas ou o destravamento total do veículo ao se interagir duas vezes com a maçaneta. 
No entanto, para fins de simplificação, este trabalho irá considerar apenas o cenário básico de destravamento de uma única porta — especificamente, aquela com a 
qual o usuário interage. Com base nisso, a seguinte história de usuário é adotada:

\begin{itemize}
    \item Como dono de um veículo automotivo, quero que a porta destrave automaticamente ao tentar abri-la com a chave em minha posse, para que eu possa acessar o veículo de forma mais rápida e conveniente.
\end{itemize}

Por fim, será implementada mais uma história de usuário, também baseada em funcionalidades descritas no manual do proprietário. Trata-se de um recurso que realiza o 
travamento automático do veículo após um determinado período de tempo, ativado em diferentes situações. Neste trabalho, será abordado especificamente o caso de 
destravamento não intencional.

Considerando que o usuário deseja manter seu veículo seguro contra furtos, garantir que ele permaneça travado é fundamental. Por esse motivo, identificar uma possível 
destrava indevida possui valor semelhante ao travamento convencional. A lógica adotada para essa funcionalidade depende de como o sistema reconhece a não intenção 
do motorista, o que deve ser inferido com base na interação do usuário com o veículo. Isso será definido na sessão do mapeamento de exemplos da história em questão, 
na próxima etapa.

Essa funcionalidade será capturada pela seguinte história de usuário:

\begin{itemize}
    \item Como dono de um veículo automotivo, quero que as portas sejam re-travadas automaticamente caso o carro seja destravado sem intenção, para que o veículo permaneça seguro contra acessos não autorizados.
\end{itemize}

\section{Etapa 2: Mapeamento de exemplos}
As histórias definidas na etapa anterior têm o papel de estabelecer um escopo claro da funcionalidade em discussão, garantindo que toda a equipe compreenda de forma 
alinhada o incremento de produto que está sendo proposto. Essa clareza é essencial para a etapa de mapeamento de exemplos, na qual ocorrem discussões colaborativas 
com pessoas de diferentes áreas.

É fundamental que todos os membros da equipe estejam cientes da ideia proposta e do valor que ela gera, pois isso permite um engajamento mais efetivo nas conversas. 
Em equipes ágeis \cite{atlassianAgileTeams}, a colaboração é um dos pilares para assegurar a qualidade do produto em desenvolvimento, especialmente em discussões que 
envolvem profissionais com diferentes formações e níveis de familiaridade técnica ou de negócio.

Para guiar a discussão, o padrão de mapeamento de exemplos utilizando notas coloridas como descrito por \cite{cucumberExampleMapping}. Com essa estrutura em mente, 
todos os pontos desconhecidos previamente identificados são registrados como perguntas, evitando que a discussão se desvie para tópicos fora do escopo naquele momento. 

As seguintes perguntas serão adicionadas em notas vermelhas na discussão da primeira história:

\begin{itemize}
    \item Como a porta é mecanicamente travada?
    \item Como a porta travada impede a sua abertura?
\end{itemize}

Após as dúvidas serem anotadas, a discussão segue focada na experiência do usuário, mesmo que alguns aspectos técnicos ainda não estejam totalmente esclarecidos.

\subsection{História de usuário 1}
Para garantir o foco na perspectiva do usuário, a discussão da história será iniciada com a listagem de exemplos concretos. assim como recomendado por 
Matt Wynne \cite{cucumberExampleMapping}, pode-se utilizar uma nomenclatura do exemplo capturado como em um episódio de seriado, que é simplificado 
no formato de \texttt{“aquele em que…”}.

Para a primeira história, alguns exemplos podem ser:

\begin{itemize}
    \item \textbf{Aquele em que o usuário estacionou o carro e travou as portas} — situação típica de uso normal, especialmente quando o veículo está em uma área onde pessoas não autorizadas podem tentar acessá-lo. Nesse caso, a intenção do usuário é evidente: as portas devem ser travadas;
    \item \textbf{Aquele em que o usuário tentou travar o veículo que já estava travado} — para o caso em que, talvez por incerteza, o botão de travamento foi pressionado mais de uma vez, para garantir que o veículo realmente estava seguro;
    \item \textbf{Aquele em que uma pessoa não autorizada tentou abrir a porta do veículo travado usando a maçaneta} — como o usuário do veículo previamente acionou o travamento, presume-se que sua intenção era impedir qualquer acesso não autorizado. Assim, neste caso, a maçaneta não deve permitir a abertura da porta.
\end{itemize}

Destes exemplos concretos, é possível extrair as seguintes regras acerca do comportamento esperado do produto:

\begin{itemize}
    \item Ao pressionar o botão de travamento, todas as portas devem ser travadas;
    \item Qualquer porta travada não deve ser aberta ao utilizar a maçaneta.
\end{itemize}

Um dos principais benefícios da metodologia de desenvolvimento orientado a comportamento já se manifesta neste estágio do processo: Os questionamentos 
provocados quando a equipe se coloca no lugar do usuário final e que surgem de forma natural em um ambiente de discussão colaborativa sobre a história 
do usuário e exemplos concretos.

Por exemplo, ao definir a primeira regra — “ao pressionar o botão de travamento, todas as portas devem ser travadas” — surge uma pergunta instintiva: 
“E se o veículo já estiver travado quando o botão for pressionado?”. Essas dúvidas enriquecem a especificação e ajudam a prever cenários reais de uso.

Neste exemplo, a história de usuário fornece uma indicação clara da intenção do usuário, expressa pelo valor que se busca alcançar. O objetivo é travar todas 
as portas, protegendo o veículo contra furtos e impedindo o acesso de pessoas não autorizadas. Diante disso, independentemente do estado inicial do veículo, 
o comportamento esperado do sistema é o de garantir o travamento completo de todas as portas.

Outras perguntas podem também surgir a partir de variações no questionamento original que foi levantado. Por exemplo: seria possível que, em determinado cenário, 
o veículo estivesse com apenas algumas portas travadas? Poderia haver um estado intermediário, em que nem todas as portas estejam travadas ou destravadas?

Como será detalhado em uma história de usuário futura, tais situações são de fato possíveis — por exemplo, quando o sistema permite o destravamento de portas 
específicas com o Keyless Access. Ainda assim, a resposta anterior permanece válida: ao acionar o botão de travamento, o sistema deve garantir que todas as portas 
estejam devidamente travadas.

A segunda regra trata da abertura da porta e estabelece que ela não deve ser permitida quando estiver travada. Esse comportamento é fundamental para a existência 
de um sistema de travamento, pois assegura que o acesso ao veículo seja restrito a pessoas autorizadas.

Embora existam incertezas técnicas quanto à forma de implementação — seja ela mecânica ou eletrônica —, define-se neste momento que, independentemente da abordagem 
adotada, uma porta travada não deve abrir quando a maçaneta for acionada. Como nos casos anteriores, as dúvidas ainda existentes serão registradas como perguntas 
adicionais para orientar futuras discussões da equipe.

Ao fim da discussão, foram capturadas 2 regras e 2 perguntas relativas a essa história de usuário. Está é a representação final das notas coloridas:

\begin{figure}[H]
\centering
\includegraphics[height=12cm]{figuras/user_story_1.png}
\caption{História de Usuário 1: Travamento de todas as portas.}
%\label{fig:casos-uso}
\end{figure}

Antes de seguir para a próxima história de usuário, aqui serão tratadas as perguntas que foram anotadas anteriormente. Esta etapa poderia ser postergada 
para o final do processo de discussão de exemplos, mas para garantir uma evolução incremental do produto, a próxima história será tratada quando os pontos 
abertos da atual forem respondidos.

Assim como descrito por \cite{reif2017locking}, sabe-se que existem 2 principais tipos de sistemas de travamento de porta que podem ser encontrados em veículos modernos:
\begin{itemize}
    \item Trava mecânica que é operada de maneira elétrica ou pneumática;
    \item Trava eletrônica que é montada com a trava e os eletrônicos embutidos.
\end{itemize}

Neste projeto, será implementado uma versão simulada de um sistema de trava eletrônica para se aproveitar de sua simplificação e menor número de componentes 
mecânicos no produto físico. Tipicamente, nesse sistema as maçanetas não precisam se mover ou podem até serem removidas inteiramente, podendo ser substituídas 
por botões equipados com sensores que se comunicam com a trava. Além disso, todos os componentes que transmitem o movimento da maçaneta no sistema mecânico aqui 
são desnecessários, substituídos pela fiação que se liga ao componente.

Embutido na trava eletrônica também é incluído o mecanismo básico de travamento como descrito por \cite{reif2017locking}. Ele é composto por três peças principais: 
o trinco (1) e engate (2) — que são acoplados à porta — e  o pino de batente (3) — que é acoplado ao chassi do veículo. Eles são responsáveis por manter a porta 
firmemente fechada, conforme ilustrado na imagem a seguir:

\begin{figure}[H]
\centering
\includegraphics[height=6cm]{figuras/trava_mecanismo.png}
\caption{Mecanismo de travamento da porta \cite{reif2017locking}.}
%\label{fig:casos-uso}
\end{figure}

Durante o fechamento da porta, o engate (2) colide com o pino de batente (3) e gira no sentido anti-horário, passando por baixo do pino até retornar à sua posição 
inicial e fazer contato com o trinco (1). Nesse ponto, a trava encontra-se na posição fechada, exatamente como demonstrado na imagem, e qualquer tentativa de abrir 
a porta força o engate no sentido horário — movimento que é impedido pela presença do trinco.

Quando a porta está destravada, no entanto, é possível abri-la ao girar o trinco no sentido anti-horário. Esse movimento libera o engate, permitindo que ele se 
afaste do pino de batente e possibilite a abertura da porta.

No caso da trava eletrônica, o acionamento desse mecanismo é controlado por lógica implementada em software e transmitido por meio de um motor que é mecanicamente 
conectado ao trinco. Assim, respondendo à segunda pergunta, a porta trava se mantém fechada ao bloquear o movimento do trinco mediante o uso da maçaneta, da mesma 
forma que a abertura é feita ao mover o trinco.

Para modelar esse comportamento do sistema, o seu escopo pode ser definido em termos de lógica de aplicação e de software básico da seguinte maneira:

\begin{itemize}
    \item \textbf{Lógica de aplicação} - gera saídas em binário para cada porta, definindo se ela está travada ou destravada e se ela deve ser aberta ou mantida fechada;
    \item \textbf{Lógica de software básico} - recebe o sinal de saída em 0 ou 1 e o converte em sinais compreensíveis para o hardware de travamento, que resultam no movimento do motor.
\end{itemize}

Além do mecanismo de travamento em si, serão utilizados botões para realizar o travamento e abertura das portas. Seu escopo será feito de maneira similar, utilizando 
valores binários como abstrações dos sinais existentes a nível de hardware que são convertidos no software básico. Em suma, para atender a primeira história de 
usuário discutida, o modelo do sistema será desenvolvido tomando como base a lógica de aplicação assim como destacado no diagrama abaixo:

\begin{figure}[H]
\centering
\includegraphics[height=8cm]{figuras/diagrama_aplication.png}
\caption{Diagrama de abstração do sistema em camadas de aplicação, software básico e componentes físicos.}
%\label{fig:casos-uso}
\end{figure}

Portanto, para a lógica de aplicação do modelo, receber o valor 1 recebido pela interface binária do botão de travamento é interpretado como um comando de 
pressionamento, da mesma forma que o valor 1 na interface do botão de abertura indica que a maçaneta foi acionada. No sentido oposto, ao enviar a saída para 
o software básico, esses valores são interpretados e convertidos em sinais de controle que determinam o movimento do motor.

Na prática, o travamento ou destravamento funciona apenas como uma limitação à abertura das portas e portanto, especialmente no caso da trava eletrônica, 
não há distinção física evidente entre os dois estados. Para facilitar essa identificação, este trabalho introduz uma indicação adicional, representada 
pela última caixa na imagem do diagrama. Ela servirá apenas para indicar o estado atual de cada porta, facilitando a execução dos testes e a validação 
funcional do sistema.

\subsection{História de usuário 2}

Ao considerar a história de usuário de destravamento das portas, é importante pensar na perspectiva da intenção de acessar o veículo. Isso é uma situação 
cotidiana que acontece com altíssima frequência, e portanto possui exemplos muito semelhantes a história 1:

\begin{itemize}
    \item \textbf{Aquele em que o usuário destravou as portas e entrou no carro} — situação típica de uso, ocorre sempre que o usuário retorna ao carro e precisa acessá-lo normalmente após o destravamento;
    \item \textbf{Aquele em que algumas portas já estavam destravadas} - caso em que o veículo se encontrava parcial ou totalmente destravado no momento em que o botão de destravamento foi acionado;
    \item \textbf{Aquele em que o usuário abriu a porta que estava destravada} —  situação em que, após o destravamento, o usuário acessa o veículo por meio da maçaneta, uma vez que a porta está destravada.
\end{itemize}

Para satisfazer os exemplos listados, as seguintes regras serão implementadas:

\begin{itemize}
    \item Ao pressionar o botão de destravamento, todas as portas devem ser destravadas.
    \item Qualquer porta destravada deve ser aberta ao utilizar a maçaneta.
\end{itemize}

Ao final, foram capturados 3 exemplos e 2 regras capturadas na história de usuário, o que compõe as seguintes notas coloridas:

\begin{figure}[H]
\centering
\includegraphics[height=12cm]{figuras/user_story_2.png}
\caption{História de Usuário 2: Destravamento de todas as portas.}
%\label{fig:casos-uso}
\end{figure}

Com base nas respostas obtidas na história anterior, diversas incertezas acerca do funcionamento do destravamento das portas já foram determinadas. Sabe-se, 
por exemplo, que a porta destravada possui uma indicação de estado e que, ao pressionar o botão de abertura, a saída do modelo deve informar ao software básico 
que o mecanismo da porta deve ser liberado — ação que será realizada por meio do sinal de controle enviado ao motor.

A única distinção nesta história é a introdução de um novo botão: o botão de destravamento. Ele exige uma interface semelhante à descrita anteriormente, sendo 
igualmente interpretada pelo software básico como um valor de tensão e convertido em um sinal binário que indica o estado atual do botão.

\subsection{História de usuário 3}

A indicação da confirmação de travamento ou destravamento do veículo tem a grande importância de permitir que o usuário esteja assegurado de que o seu comando foi 
recebido e executado pelo veículo. Ela deve portanto ser invocada sempre que o usuário estiver utilizando as histórias mencionadas anteriormente, além de capturar 
os seguintes exemplos:

\begin{itemize}
    \item \textbf{Aquele em que o usuário travou seu veículo e ficou na dúvida se as portas foram travadas ou destravadas} - caso em que o usuário utiliza o botão de travamento e espera que o seu veículo seja assegurado. Deve invocar uma indicação distinta da indicação de destravamento;
    \item \textbf{Aquele em que o usuário destravou seu veículo e ficou na dúvida se as portas foram travadas ou destravadas} - caso em que o usuário utiliza o botão de destravamento e espera que o seu veículo seja destravado. Deve invocar uma indicação distinta da indicação de travamento;
    \item \textbf{Aquele em que o usuário tentou travar seu veículo sem perceber que uma das portas ficou aberta} - pode acontecer quando uma ou mais portas ficam abertas por conta de alguém esquecer de fechá-la. Neste caso, mesmo travando a porta que está aberta, o veículo ainda não fica seguro contra o acesso indevido e portanto deve invocar uma indicação de falha.
\end{itemize} 

As regras que cobrem os exemplos listados são:

\begin{itemize}
    \item O veículo deve indicar que foi travado ou destravado com sucesso utilizando sinais distintos para cada comando;
    \item O veículo deve indicar com um sinal de falha quando pelo menos uma das portas está aberta ao receber o comando de travamento.
\end{itemize}

O último exemplo levanta questões que precisam ser exploradas por meio de perguntas e trazem consigo um novo fator que o sistema deve considerar: O estado de 
abertura das portas. Para isso, as seguintes perguntas são criadas:

\begin{itemize}
    \item Como o sistema detecta se uma porta está aberta ou fechada?
    \item É possível travar ou destravar uma porta que está aberta?
\end{itemize}

Dessa forma, essa história de usuário foi mapeada com 3 exemplos, 2 regras e 2 perguntas, da seguinte maneira:

\begin{figure}[H]
\centering
\includegraphics[height=12cm]{figuras/user_story_3.png}
\caption{História de Usuário 3: Feedback de travamento.}
%\label{fig:casos-uso}
\end{figure}

A investigação das perguntas levantadas envolvem detalhes que dependem diretamente da implementação física do produto. Para respondê-las, é necessário 
compreender como um sistema veicular real consegue identificar se as portas estão fisicamente abertas.

Esta detecção é normalmente feita por meio de sensores ou mecanismos capazes de indicar quando a porta está entreaberta (estado ajar). Um exemplo é 
apresentado em \cite{us4566285a}, que descreve o uso de switches físicos: eles são pressionados quando a porta está completamente fechada e liberados 
quando a porta se abre. 

Esse tipo de sensor pode ser integrado de forma semelhante às interfaces binárias utilizadas nos botões de travamento e destravamento, empregando sinais 
binários (0 ou 1) para representar os estados "fechada" ou "aberta" da porta. Com essa informação, o sistema pode combinar o estado físico da porta com a 
entrada da interface do botão de travamento para determinar quando o alerta para o usuário deve ser realizado.

É importante destacar que, embora o controle de abertura da porta seja gerenciado pelo sistema, a utilização de um sensor continua sendo essencial para 
determinar seu estado real. Esse cenário é comum no controle de componentes mecânicos, em que parte do comportamento não é governada diretamente pela 
lógica de software ou eletrônica.

No caso em questão, a saída final do sistema de travamento consiste no acionamento do motor que movimenta o trinco e libera o mecanismo da trava, permitindo 
que a porta seja aberta. No entanto, uma vez que essa ação é executada, o sistema não tem como saber se o usuário realmente abriu a porta — ou se, após abri-la, 
voltou a fechá-la — sem o auxílio de um sensor. Isso ocorre porque a abertura da porta é uma ação mecânica realizada pelo usuário, sendo apenas condicionada, e 
não diretamente executada, pelo sistema eletrônico em desenvolvimento.

Além disso, o travamento e destravamento das portas é realizado de maneira independente do estado da porta e pode ser realizado mesmo que ela esteja aberta. Isso 
deve-se ao fato de que o controlador eletrônico é responsável por definir quando o mecanismo da trava deve ser acionado. Caso a porta esteja destravada, esta ação 
é tomada após o pressionar do botão de abertura da porta.

Em termos práticos, isso significa que não existe diferença física entre uma porta travada ou destravada, este é apenas um estado que é virtualmente definido 
pelo controlador. É importante ressaltar que travar uma porta aberta ainda não satisfaz a condição de assegurar o veículo, como definido na primeira história 
de usuário e por conta disso o alerta do usuário se vê necessário.

\subsection{História de usuário 4}

A funcionalidade de Keyless Access permite o destravamento individual das portas assim que o usuário aciona a maçaneta de uma porta travada, desde que a chave 
autorizada esteja presente em sua posse. Os exemplos de uso dessa funcionalidade são bastante semelhantes aos do destravamento convencional, pois refletem 
situações em que o usuário tem a intenção de acessar o veículo. 

Neste estágio, ainda não foi definida a tecnologia que será utilizada para a identificação da chave. Por isso, essa incerteza será tratada por meio de uma 
pergunta associada a esta história de usuário.

Os exemplos capturados na discussão da história são:

\begin{itemize}
    \item \textbf{Aquele em que o usuário tinha a chave no bolso e usou a maçaneta} — Este é o caso típico de uso da funcionalidade: a chave autorizada está próxima e é reconhecida pelo sistema, permitindo o destravamento imediato da porta acionada;
    \item \textbf{Aquele em que outra pessoa abriu outra porta simultaneamente} — Neste caso, mais uma porta é aberta simultaneamente, possivelmente por alguém acompanhando o dono do veículo. O sistema deve tratar as portas de forma independente, e portanto ambas devem ser destravadas individualmente, permitindo o acesso simultâneo;
    \item \textbf{Aquele em que o usuário destravou a porta e a fechou de novo} — Após acessar o veículo utilizando o Keyless Access e em seguida fechar a porta, o sistema deve manter a porta destravada até que o usuário execute explicitamente o comando de travamento. Isso garante que a intenção do usuário seja respeitada e evita travamentos não intencionais.
\end{itemize}

Para cobrir os exemplos, as seguintes regras serão utilizadas:

\begin{itemize}
    \item Qualquer porta que tiver sua maçaneta utilizada deve ser destravada desde que a chave esteja próxima ao veículo;
    \item A porta deve ser mantida destravada após a operação de Keyless Access.
\end{itemize}

O mapeamento de exemplos dessa história na forma de notas coloridas é composto por 3 exemplos, 2 regras e 1 pergunta:

\begin{figure}[H]
\centering
\includegraphics[height=12cm]{figuras/user_story_4.png}
\caption{História de Usuário 4: Keyless Access.}
%\label{fig:casos-uso}
\end{figure}

Por fim, uma nova pergunta foi levantada no mapeamento de exemplos da história e deve ser respondida antes do trabalho na próxima história. Para fazê-lo, é 
preciso investigar alguma opção de tecnologia que possa ser utilizada para representar a validação de chave de usuário. 

A implementação de sistemas de detecção de chave, conforme descrito por \cite{glocker2016protocol}, costumam utilizar módulos de comunicação por rádio-frequência (RF), 
capazes de estabelecer conexão com um dispositivo autorizado. A validação desse dispositivo pode ser realizada por meio de diferentes métodos e protocolos de segurança, 
que incluem mecanismos de autenticação mútua entre a chave e o veículo. Esses protocolos aumentam a robustez do sistema contra tentativas de acesso não autorizado.

Em casos mais simples, como demonstrado em \cite{arduinoRFID}, o sistema utiliza um sensor RFID (Radio Frequency Identification) e uma tag que contém uma chave 
de identificação única. O funcionamento baseia-se na transmissão do identificador por meio de um sinal de rádio quando a tag é aproximada do sensor. O sensor, 
por sua vez, envia o código recebido ao microcontrolador, que o compara com um identificador previamente armazenado em sua memória. Caso haja correspondência, 
a chave é considerada válida e o acesso é autorizado.

Esse tipo de autenticação, no contexto da arquitetura AUTOSAR, é normalmente responsabilidade de um componente específico do software básico pertencente 
à pilha de cibersegurança. Esse componente, conhecido como Key Manager \cite{autosarKeyManager}, fornece serviços utilizados para a validação de chaves, 
utilizando diferentes protocolos de segurança. Ele também se comunica com a unidade de memória do controlador para acessar os dados das chaves previamente registradas.

Considerando a presença desse componente no sistema AUTOSAR, será adotada novamente uma abstração da tecnologia de hardware envolvida, por meio de uma interface 
binária. Nesse cenário, a verificação da chave de identificação em relação ao valor armazenado em memória é realizada pelo Key Manager e o resultado dessa 
validação é então comunicado à aplicação de forma semelhante ao funcionamento dos botões de entrada: por meio de um sinal binário que indica se a chave foi 
validada (1) ou não (0).

\subsection{História de usuário 5}
A funcionalidade de auto travamento é comumente implementada para contemplar diferentes cenários de uso do veículo. Neste caso específico, o foco está na situação em 
que as portas são destravadas de forma não intencional. Antes de apresentar os exemplos, será feita uma definição do que se entende por "não intenção" por parte do 
usuário, considerando que, na prática, essa intenção precisa ser inferida com base nas entradas disponíveis no sistema. 

Esse tipo de questão costuma ser capturado em perguntas, nas notas vermelhas, para que possa ser investigado após o mapeamento de exemplos. No entanto, para essa 
história é essencial que a definição de não intencional seja feita, pois ela definirá com clareza que exemplos de uso devem ser mapeados.

De modo geral, o destravamento das portas tem como objetivo permitir o acesso ao veículo, o que se concretiza quando o usuário abre pelo menos uma das portas. Utilizando 
esse critério, a ausência da abertura das portas dentro de um determinado intervalo de tempo após o destravamento pode ser interpretada como um indicativo de que o 
usuário não teve a intenção consciente de destravar o veículo. 

Essa condição também é presente em veículos Volkswagen, assim como descrito no manual do usuário \cite{vwLocking}. Segundo ele, o veículo realiza o travamento automático 
em até 45 segundos após ter sido destravado, desde que, entre outras possíveis condições, nenhuma das portas tenha sido aberta. A fim de tornar a ação de travamento 
automático mais rápida e facilitar a execução dos testes, o tempo de espera para essa condição será reduzido para 15 segundos neste trabalho.

Adicionalmente, o estado inicial das portas no momento do destravamento também é de relevância para esse comportamento. Nos casos em que o veículo não estava 
completamente travado previamente, infere-se que o usuário não possuía a intenção de garantir a segurança do veículo, mesmo antes do botão de destravamento ser 
pressionado, e portanto ele não precisa ser travado novamente.

Sabe-se que o principal objetivo do travamento do veículo é garantir sua segurança contra furtos e impedir o acesso de pessoas não autorizadas. No entanto, existem 
cenários em que o usuário pode intencionalmente optar por deixar o veículo destravado, mesmo que não pretenda acessá-lo de imediato. Um exemplo disso seria quando 
o veículo é deixado em uma garagem particular, onde o usuário deseja mantê-lo acessível para buscar um item posteriormente ou por qualquer outro motivo, sem a 
necessidade de utilizar novamente a chave ou o sistema de Keyless Access.

Para contemplar essa situação e permitir que a intenção do usuário seja respeitada, é necessário que exista pelo menos uma regra específica que permita o 
destravamento completo, sem que seja necessário o acesso ao veículo. Neste caso, será adotada a estratégia de considerar o duplo acionamento do botão de 
destravamento como uma indicação explícita da intenção de manter o veículo destravado.

Por fim, os exemplos capturados na história são:

\begin{itemize}
    \item \textbf{Aquele em que o usuário destravou mas não foi acessar o carro} — este é o cenário padrão de destravamento não intencional, que pode ocorrer por diversos motivos, como o acionamento acidental do botão de destravamento. Nessa situação, o sistema realizará o travamento automático após um intervalo de tempo pré-definido, interpretando que não houve intenção de acesso;
    \item \textbf{Aquele em que o usuário destravou e foi acessar o carro} — neste cenário, a situação oposta à do exemplo anterior acontece, pois o usuário acessou o veículo, abrindo uma das portas, em tempo. O travamento automático não deve ser acionado neste caso;
    \item \textbf{Aquele em que o usuário pressionou o botão de destravamento duas vezes para manter o carro destravado sem precisar abrir uma porta} — Quando for da intenção do usuário deixar o veículo destravado sem abrir nenhuma porta, a função de travamento automático será inibida mediante o duplo acionamento do botão de destravamento. Essa ação será interpretada como uma indicação explícita da intenção de manter o veículo acessível;
    \item \textbf{Aquele em que o usuário destravou o veículo quando ele já estava acessível} — refere-se às situações em que o veículo não estava completamente travado inicialmente, seja porque uma ou mais portas já estavam destravadas ou por conta do acionamento repetido do botão de destravamento. Nestes casos, o sistema entende que o usuário já interagiu com o veículo e portanto o travamento automático também não será executado.
\end{itemize}

Com os exemplos mapeados, o comportamento definido garante que o travamento automático trata em específico os casos em que a ação do usuário indicou que ele não possuía a intenção de mudar o veículo de um estado inicial em que ele está absolutamente seguro contra furtos - todas as portas estavam travadas - para o caso em que qualquer pessoa poderia acessá-lo. As regras que satisfazem os exemplos são:

\begin{itemize}
    \item Travar automaticamente caso o botão de destravamento seja usado uma vez e nenhuma porta seja aberta;
    \item Não travar caso o botão de destravamento seja pressionado uma segunda vez;
    \item Não travar caso o veículo não esteja completamente travado antes do destravamento.
\end{itemize}

O mapeamento de exemplos dessa história na forma de notas coloridas é composto por 4 exemplos e 3 regras:

\begin{figure}[H]
\centering
\includegraphics[height=12cm]{figuras/user_story_5.png}
\caption{História de Usuário 5: Travamento automático.}
%\label{fig:casos-uso}
\end{figure}


\section{Etapa 3: Desenho do diagrama de caixa preta}
Com todas as histórias de usuário definidas, o próximo passo do trabalho consiste na elaboração do diagrama de caixa preta, com o objetivo de identificar todas as 
interfaces necessárias para suportar os comportamentos previamente levantados. Essa etapa é essencial para garantir que todas as interações entre o usuário e o 
sistema estejam claramente especificadas e para suportar a criação dos cenários Gherkin a seguir.

Para dar início ao processo, é necessário listar todas as entradas e saídas do sistema:

\textbf{Entradas}:

\begin{itemize}
    \item Botão de travamento
    \item Botão de destravamento
    \item Botões de abertura das portas (4 no total)
    \item Sensores de abertura das portas (4 no total)
    \item Detecção de chave autenticada.
\end{itemize}

\textbf{Saídas}:

\begin{itemize}
    \item Indicação de estado de travamento de todas as portas (4 no total)
    \item Estado da tranca de cada porta (4 no total)
    \item Feedback para o usuário.
\end{itemize}

O diagrama é então elaborado com um bloco central representando o sistema, ao qual estão conectadas todas as interfaces previamente listadas. Neste estágio, 
o bloco central permanece sem conteúdo interno, uma vez que o objetivo é abstrair os detalhes da implementação e focar exclusivamente na definição das 
interações entre o sistema e o usuário.

Dessa forma, o diagrama assume a seguinte configuração:

\begin{figure}[H]
\centering
\includegraphics[height=8cm]{figuras/diagrama_caixa_preta.png}
\caption{Diagrama de Caixa Preta do sistema de travamento de portas.}
%\label{fig:casos-uso}
\end{figure}

De maneira análoga, o modelo desenvolvido no Simulink também deverá adotar uma estrutura semelhante, composta por um modelo de alto nível — que define apenas as 
entradas e saídas do sistema — e um modelo de baixo nível, responsável por conter toda a lógica funcional implementada.

É importante destacar que algumas interfaces adicionais serão necessárias no modelo do sistema, embora não estejam representadas no diagrama de caixa preta, 
pois não estão diretamente relacionadas aos comportamentos observáveis pelo usuário. Essas interfaces serão detalhadas posteriormente, durante a etapa de 
modelagem iterativa, e são fundamentais para suportar aspectos técnicos que garantem o funcionamento correto do sistema. Um exemplo dessas interfaces é o tempo 
de simulação, necessário para validar a condição de auto travamento descrita na história de usuário 5.

Para finalizar essa etapa, é necessário definir quais são os valores possíveis que devem ser capturados em cada uma das interfaces criadas neste diagrama:

% \begin{tabular}{|p{5cm}|p{7cm}|}
% %\centering
% %\caption{Valores possíveis para as interfaces de entrada e saída}
% %\addcontentsline{loq}{table}{\protect\numberline{\thetable}Valores possíveis para as interfaces de entrada e saída}
% \hline
% Nome da interface 1 & Valores Possíveis \\ 
% \hline
% Botão de travamento geral & 
% \begin{itemize}[topsep=0pt, partopsep=0pt, leftmargin=*]
%     \item Pressionado
%     \item Solto
% \end{itemize} \\
% \hline
% Botão de destravamento geral &
% \begin{itemize}[topsep=0pt, partopsep=0pt, leftmargin=*]
%     \item Pressionado
%     \item Solto
% \end{itemize} \\
% \hline
% Detecção de chave &
% \begin{itemize}[topsep=0pt, partopsep=0pt, leftmargin=*]
%     \item Presente
%     \item Não presente
% \end{itemize} \\
% \hline
% Botão de abetura de porta (de 1 a 4) &
% \begin{itemize}[topsep=0pt, partopsep=0pt, leftmargin=*]
%     \item Pressionado
%     \item Solto
% \end{itemize} \\
% \hline
% Sensor de porta aberta (de 1 a 4) &
% \begin{itemize}[topsep=0pt, partopsep=0pt, leftmargin=*]
%     \item Fechada
%     \item Aberta
% \end{itemize} \\
% \hline
% Indicação de estado da porta (de 1 a 4) &
% \begin{itemize}[topsep=0pt, partopsep=0pt, leftmargin=*]
%     \item Travada
%     \item Destravada
% \end{itemize} \\
% \hline
% Tranca da porta 1 &
% \begin{itemize}[topsep=0pt, partopsep=0pt, leftmargin=*]
%     \item Segura
%     \item Solta
% \end{itemize} \\
% \hline
% Feedback de travamento &
% \begin{itemize}[topsep=0pt, partopsep=0pt, leftmargin=*]
%     \item Confirmação de trava
%     \item Confirmação de destrava
%     \item Operação falha
% \end{itemize} \\
% \hline
% %\label{qua:interfaces}
% \end{tabular}

\begin{quadro}[h]
\caption{Valores possíveis para as interfaces de entrada e saída}
\label{qua:interfaces}
\begin{tabular}{|p{5cm}|p{7cm}|}
\hline
Nome da interface 1 & Valores Possíveis \\ 
\hline
Botão de travamento geral & 
\begin{itemize}[topsep=0pt, partopsep=0pt, leftmargin=*]
    \item Pressionado
    \item Solto
\end{itemize} \\
\hline
Botão de destravamento geral &
\begin{itemize}[topsep=0pt, partopsep=0pt, leftmargin=*]
    \item Pressionado
    \item Solto
\end{itemize} \\
\hline
Detecção de chave &
\begin{itemize}[topsep=0pt, partopsep=0pt, leftmargin=*]
    \item Presente
    \item Não presente
\end{itemize} \\
\hline
Botão de abertura de porta (de 1 a 4) &
\begin{itemize}[topsep=0pt, partopsep=0pt, leftmargin=*]
    \item Pressionado
    \item Solto
\end{itemize} \\
\hline
Sensor de porta aberta (de 1 a 4) &
\begin{itemize}[topsep=0pt, partopsep=0pt, leftmargin=*]
    \item Fechada
    \item Aberta
\end{itemize} \\
\hline
Indicação de estado da porta (de 1 a 4) &
\begin{itemize}[topsep=0pt, partopsep=0pt, leftmargin=*]
    \item Travada
    \item Destravada
\end{itemize} \\
\hline
Tranca da porta 1 &
\begin{itemize}[topsep=0pt, partopsep=0pt, leftmargin=*]
    \item Segura
    \item Solta
\end{itemize} \\
\hline
Feedback de travamento &
\begin{itemize}[topsep=0pt, partopsep=0pt, leftmargin=*]
    \item Confirmação de trava
    \item Confirmação de destrava
    \item Operação falha
\end{itemize} \\
\hline
\end{tabular}
\end{quadro}

Nesta tabela, os valores das interfaces foram descritos de maneira textual, com valores que se aplicam à linguagem natural proposta pela metodologia de BDD. 
Na prática, todos os valores serão utilizados como enumerações convertidas para um valor numérico que representa o estado da interface.

\section{Etapa 4: Desenvolvimento dos cenários Gherkin}
\subsection{História de usuário 1}
\subsection{História de usuário 2}
\subsection{História de usuário 3}
\subsection{História de usuário 4}
\subsection{História de usuário 5}
\section{Etapa 5: Modelagem iterativa do sistema em simulink}
